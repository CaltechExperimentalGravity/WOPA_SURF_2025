\documentclass[colorlinks=true,pdfstartview=FitV,linkcolor=blue,
citecolor=red,urlcolor=magenta]{ligodoc}

\usepackage{siunitx}
\usepackage{graphicx}
\usepackage{amssymb}
\usepackage{amsmath}
\usepackage{longtable}
\usepackage{rotating}
\usepackage[usenames,dvipsnames]{color}
\usepackage{fancyhdr}
\usepackage{subfigure}
\usepackage{hyperref}

\ligodccnumber{T}{22}{xxxxx}{}{}
\title{}
\author{}
\begin{document}

\tableofcontents

\newpage
\section{Introduction}
Introduce LIGO and bring in how your project contributes to LIGO research.

LIGO (Laser Interferometer Gravitational-Wave Observatory) detects gravitational waves from astronomical phenomena such as black holes and binary neutron star collisions by using interferometry to measure the movement of test masses as a result of a gravitational wave. LIGO consists of two L-shaped interferometers located ~3000 km apart whose arms are orthogonal 4 km Michelson interferometers. A 1064nm wavelength laser is split at a beamsplitter and sent down the two arms. When a gravitational wave passes, it will stretch one of the arms while compressing another, resulting in a phase difference of the beams in each arm. When the beams recombine at the beamsplitter the phase shift of the beam can be detected at a photodetector. Two interferometers are used to rule out local noise that could have caused the phase shift. The level of sensitivity required to detect gravitational waves also makes LIGO's measurements susceptible to noise. To optimize the accuracy and sensitivity of LIGO, we aim to limit noise from as many sources as possible. One way noise is minimized in the laser is through quantum squeezing. 
\section{Objective}
Set goals for yourself here.

\section{Approach}
Describe a working plan. Cite an article like \cite{PDH}.

\section{Timeline}
Describe timeline of your plan, explaining what you would do every 2 weeks.

\begin{thebibliography}{9}
\bibitem{PDH} Authors. \emph{Paper Title}. Journal Abbreviation, Vol, Page numbers (YYYY).
\bibitem{controls} Authors. \emph{Website Title}. \url{link}.

\end{thebibliography}
\end{document}
