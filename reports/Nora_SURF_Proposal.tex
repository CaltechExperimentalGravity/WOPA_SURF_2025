\documentclass[colorlinks=true,pdfstartview=FitV,linkcolor=blue,
citecolor=red,urlcolor=magenta]{ligodoc}

\usepackage{siunitx}
\usepackage{graphicx}
\usepackage{amssymb}
\usepackage{amsmath}
\usepackage{longtable}
\usepackage{rotating}
\usepackage[usenames,dvipsnames]{color}
\usepackage{fancyhdr}
\usepackage{subfigure}
\usepackage{hyperref}

\ligodccnumber{T}{22}{xxxxx}{}{}
\title{Minimizing Noise in WOPA (Waveguided Optical Parametric Amplification) by Optimizing Mode Matching}
\author{Rana Adhikari, Peter G. Carney, Nora Dreslin}
\begin{document}

\tableofcontents

\newpage
\section{Introduction}
LIGO (Laser Interferometer Gravitational-Wave Observatory) detects gravitational waves from astronomical phenomena such as black holes and binary neutron star collisions by using interferometry to measure the movement of test masses as a result of a gravitational wave. LIGO consists of two L-shaped interferometers located ~3000 km apart whose arms are orthogonal 4 km Michelson interferometers. A 1064nm wavelength laser is split at a beamsplitter and sent down the arms so that when a gravitational wave passes and sqeezes one arm while compressing the other, the beams will have a phase difference. The level of sensitivity required to detect gravitational waves (strain of gravitational wave is on the order of $10^{-22}$) also makes LIGO's measurements susceptible to noise. To optimize the accuracy and sensitivity of LIGO, we aim to limit noise from as many sources as possible. One way noise is minimized in the laser is through quantum squeezing. 

Quantum squeezing is a method in which we can minimize the laser shot noise. This shot noise arises from the uncertainty between phase and amplitude. We are able to reduce the uncertainty in phase by increasing the uncertainty in amplitude and vice versa. Currently, both amplitude and phase squeezing is being used to reduce noise in LIGO, depending on the frequency. This project aims to improves on the squeezing process by using a periodically poled lithium niobate (PPLN) crystal to generate 1064nm squeezed light from 532nm wavelength light which we can combine with the 1064nm signal in LIGO to squeeze the light being detected. 
\section{Objective}
Within the squeezing process itself there are sources of noise. One of which is a result of mode mismatch between the squeezed beam coming from the crystal and the non-squeezed beam. Loss of mode matching results in loss of measured squeezed light, given by
$$dB_{measured} = 10\log((1-\eta) + (\eta)10^{(dB/10)})$$
Where $\eta$ is the mode matching with 1 being perfectly mode matched. 0.95\% mode matching only allows for measurement of a maximum 99.77\% of actual squeezing. This project aims to measure and minimize noise as a result of mode mismatch between squeezed and non-squeezed beams to optimize the accuracy and sensitivity of LIGO detections.
We hope to mode match our local oscillator (LO) beam and signal beam to within $\eta=0.97$ and generate 6dB+ of squeezing.

\section{Approach}
To maximize the measured dB of squeezing, we will focus on optimizing the mode matching between the signal and LO. Using ABCD matrices and FINESSE, an interferometer simulation program, we will model the LO and signal beams as they propagate through the setup shown above and adjust the WOPA setup until the beams are mode matched in the TEM00 mode. We match the beams by aligning their profiles and the positions of the beam waists which are given by
$$w(z)=w_0\sqrt{1+\left(\frac{z}{z_R}\right)^2}$$
where $w_0$ is the waist radius and $z_R$ is the Rayleigh range. 
\section{Timeline}
\begin{center}
\begin{tabular}{|c|m{15cm}|}
    \hline
    Week 1 (6/18-6/25) & Measure the laser polarization drift, a possible source of noise in the system. We will also model the mode of the beam coming into and out of the crystal.\\
   \hline
    Weeks 2-3 (6/25-7/9) & Take nonlinear gain measurements of the laser.\\
    \hline
    Weeks 4-5 (7/9-7/23) & Measure the squeezing of light. \\
    \hline
    Weeks 6-7 (7/23-8/6) & Mode matching our beams and reorienting the setup.\\
    \hline
    Weeks 8-9 (8/6-8/20) & Coherent phase locking the system to optimize the squeezing measurement\\
    \hline
\end{tabular}
\end{center}    
\begin{thebibliography}{9}
\bibitem{LIGO} Barry C. Barish and Rainer Weiss. \emph{LIGO and the Detection of Gravitational Waves}. Physics Today, 52, p. 44-50 (1999).
\bibitem{fiber sqz} F. Kaiser et. al. \emph{A Fully Guided-Wave Squeezing Experiment for Fiber Quantum Networks}. Optica, Vol. 3, p. 362 (2016).

\end{thebibliography}
\end{document}
