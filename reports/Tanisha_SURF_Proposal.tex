\documentclass[colorlinks=true,pdfstartview=FitV,linkcolor=blue,
citecolor=red,urlcolor=magenta]{ligodoc}

\usepackage{siunitx}
\usepackage{graphicx}
\usepackage{amssymb}
\usepackage{amsmath}
\usepackage{longtable}
\usepackage{rotating}
\usepackage[usenames,dvipsnames]{color}
\usepackage{fancyhdr}
\usepackage{subfigure}
\usepackage{hyperref}
\usepackage{float}
\ligodccnumber{T}{22}{xxxxx}{}{}
\title{\large{ Noise Characterisation and Suppression in a Waveguided Optical Parametric Amplification (WOPA) -Based Quantum Squeezing Experiment}}
\author{}
\begin{document}

\tableofcontents

\newpage
\section{Introduction}
% Introduce LIGO and bring in how your project contributes to LIGO research.
The Laser Interferometer Gravitational-Wave Observatory (LIGO) detects gravitational waves by measuring incredibly small distortions in spacetime, with a strain on the order of \(10^{-22}\) m/m, utilizing highly sensitive laser interferometers. One significant limitation to LIGO's sensitivity is shot noise, a form of quantum noise that hampers performance. At low frequencies, this manifests as radiation pressure noise on the mirrors, while at high frequencies, it appears as frequency or phase noise, which restricts detection capabilities. Squeezed light injection — a technique that reduces shot noise — is thus vital to LIGO's performance enhancements.

Currently, LIGO achieves approximately 4 dB of squeezing, and our goal is to produce greater amounts in a more efficient manner, such as through single-pass methods without cavities. Our technique may also be applicable for other quantum communication applications. Our project contributes to this initiative by developing a detailed noise budget analysis for a Waveguided Optical Parametric Amplification (WOPA) experiment. In this tabletop experimental setup, several decibels of quantum squeezing at \SI{1064}{\nano\meter} are generated by pumping a Periodically Poled Lithium Niobate (PPLN) waveguide with \SI{532}{\nano\meter} light. The squeezed \SI{1064}{\nano\meter} beam is then combined with a local oscillator (LO) to prepare it for eventual injection into an interferometer similar to those used in LIGO. Calculating the noise budget for frequency noise and phase noise in this system is essential to ensuring that we meet the squeezing levels ($> 6 dB$) required for substantial improvements in LIGO’s sensitivity in the future.

\section{Objective}
% Set goals for yourself here.
The main goals of this project are:

\begin{itemize}
    \item To calculate and quantify the noise budget, particularly focusing on frequency noise and phase noise contributions, for the WOPA setup described.  Additionally, to address and quantify other sources of noise such as polarisation mismatch, gain imbalance at the balanced homodyne detector (BHD), mode-matching errors between the local oscillator (LO) and signal fields, and residual intensity noise.
    \item To establish tolerances for phase stability that ensure more than \SI{6}{\decibel} of observed squeezing.
    %\item To provide practical recommendations to mitigate these noise sources for the successful integration of the squeezed source into an interferometric gravitational wave detector like LIGO.
    \item To determine the necessary length scale for suppressing laser frequency noise,  given the current frequency variation of 75 MHz. This will involve quantifying how frequency noise translates into path length variations and setting requirements for laser stabilisation to meet stringent interferometric precision.
\end{itemize}

%\begin{figure}[H]
   % \centering
   % \includegraphics[width=0.8\textwidth]{C:/Users/Tanisha Ray/OneDrive/Desktop/LIGO_Ch/WOPA_SURF_2025/reports/figures/WOPA_Set_Up.png}
    %\caption{WOPA Experimental Setup}
    %\label{fig:example}
%\end{figure}

In a homodyne detection setup such as the WOPA configuration, the signal beam (carrying squeezed vacuum) and the local oscillator (LO) beam traverse distinct arms of a Mach-Zehnder-type interferometer before recombining at a beamsplitter. When these two arms are not precisely matched in length, the setup becomes susceptible to laser frequency noise, which introduces a time-dependent phase fluctuation into the readout and thereby degrades the observable squeezing.

Let the electric fields of the local oscillator and the squeezed signal be given by:
\begin{align}
    E_{\text{LO}}(t) &= E_0 \, e^{-i[\omega t + \phi_{\text{LO}}(t)]} \\
    E_{\text{sig}}(t) &= E_0' \, e^{-i[\omega t + \phi_{\text{sig}}(t)]}
\end{align}

Let the mismatch in optical path lengths between the LO and signal arms be \( \Delta L \). The phase accumulated by each beam over its respective path length depends on the instantaneous optical frequency \( \nu(t) = \nu_0 + \delta \nu(t) \), where \( \delta \nu(t) \) is the laser frequency noise. The corresponding wavenumber is:
\begin{equation}
    k(t) = \frac{2\pi \nu(t)}{c} = \frac{2\pi}{c} (\nu_0 + \delta \nu(t))
\end{equation}

Thus, the time-dependent phase acquired along a path of length \( L \) is:
\begin{equation}
    \phi(t) = k(t) L = \frac{2\pi}{c} \nu(t) L = \underbrace{\frac{2\pi \nu_0 L}{c}}_{\text{static}} + \underbrace{\frac{2\pi \delta \nu(t) L}{c}}_{\text{dynamic}}
\end{equation}

The phase difference between the two arms is then:
\begin{equation}
    \delta \phi(t) = \phi_{\text{LO}}(t) - \phi_{\text{sig}}(t) = \frac{2\pi \nu_0 \Delta L}{c} + \frac{2\pi \delta \nu(t) \Delta L}{c}
\end{equation}

Here, the first term is a static offset, which can be compensated via phase control. The second term is a fluctuating phase shift due to laser frequency noise:
\begin{equation}
    \delta \phi(t) \approx \frac{2\pi \Delta L}{c} \delta \nu(t)
\end{equation}

In our system, the laser exhibits approximately 75~MHz of frequency noise, i.e., \( \delta \nu \approx 75 \times 10^6 \)~Hz. This phase noise rotates the measurement quadrature and mixes the anti-squeezed component into the detected squeezed quadrature, degrading the observed squeezing level.

The measured squeezing level \( S_{\text{meas}} \) (in dB) in the presence of small phase noise \( \delta \phi \) is approximately:
\begin{equation}
    S_{\text{meas}} \approx -10 \log_{10} \left( e^{-2r} \cos^2 \delta \phi + e^{2r} \sin^2 \delta \phi \right)
\end{equation}
where \( r \) is the squeezing parameter, and the ideal squeezing level (in dB) is:
\begin{equation}
    S_{\text{ideal}} = -10 \log_{10}(e^{-2r})
\end{equation}

To observe more than 6~dB of squeezing (corresponding to \( r \approx 0.69 \), \( e^{2r} \approx 4 \), and \( e^{-2r} \approx 0.25 \)), the phase fluctuation must satisfy \( \delta \phi \lesssim 0.14 \)~rad to limit degradation to under 1~dB. Substituting into the phase noise expression:
\begin{align}
    \Delta L &\lesssim \frac{\delta \phi_{\text{max}} \cdot c}{2\pi \, \delta \nu} \\
             &\lesssim \frac{0.14 \cdot 3 \times 10^8 \, \text{m/s}}{2\pi \cdot 75 \times 10^6 \, \text{Hz}} \\
             &\lesssim 0.089 \, \text{m} \approx 8.9 \, \text{cm}
\end{align}

\noindent\textbf{Conclusion:} To observe more than 6~dB of squeezing in the presence of 75~MHz laser frequency noise, the optical path length difference between the LO and signal arms must be maintained below approximately \textbf{9~cm}. Minimising this mismatch is essential to suppress phase noise and preserve the integrity of the squeezed state at the readout.



%\section{Approach}
% Describe a working plan. Cite an article like \cite{PDH}.


%\begin{itemize}
    %\item Model the squeezed light source using the theoretical framework of quantum optics, particularly the impact of frequency noise and phase noise on the squeezed quadratures.
    %\item Analyse the WOPA optical setup, including the PPLN waveguide, LO phase shifter, and detection optics, using parameters extracted from the experimental layout.
   % \item Develop noise transfer functions that describe how laser frequency noise and LO phase noise couple into the measured squeezed quadrature.
    %\item Simulate the noise degradation using the mathematical relationship between squeezing, anti-squeezing, and phase noise.
    %\item Validate noise budgets experimentally against squeezing degradation thresholds, setting allowable noise levels to meet $>\SI{6}{\decibel}$ observed squeezing.
   % \item Propose methods to control phase noise, such as active feedback stabilization and passive noise reduction.
%\end{itemize}

\section{Approach}
The project will be conducted in the following manner:
\begin{itemize}
    \item The initial phase of the project will focus on familiarisation with the WOPA experimental setup and identification of key optical and electronic parameters from the layout. This includes analysing the PPLN waveguide, LO phase shifter, and balanced homodyne detection optics. The primary objective is to reduce the impact of laser frequency noise on the measured squeezing. To achieve this, the optical paths of the LO and signal arms will be carefully examined and adjusted. Optical elements (mirrors, lenses, fibres) will be repositioned to reduce the path length difference $\Delta L$ to below the calculated threshold of 9~cm. This alignment is critical to prevent excessive phase fluctuations from entering the readout. Measurements will be taken with a frequency discriminator and spectrum analyser to quantify the laser frequency noise spectrum, typically around 75~MHz, and confirm that its coupling into the squeezed quadrature is within tolerable limits.

    \item The squeezed light source will be modelled using quantum optics formalism, specifically considering the variances of the squeezed and anti-squeezed quadratures and how they are affected by frequency and phase noise. Using the relationship $\delta \phi = 2\pi \Delta L \cdot \delta \nu / c$, where $\delta \nu$ represents frequency noise and $\Delta L$ the path mismatch, simulations will be performed to estimate the squeezing degradation as a function of arm mismatch and frequency noise amplitude.

    \item Transfer functions will be derived and simulated to describe how both laser frequency noise and LO phase fluctuations couple into the measured quadratures of the homodyne detector. These models will be used to calculate the RMS phase noise levels allowable for squeezing greater than 6~dB, and the results will be cross-checked against experimental squeezing measurements under different optical alignment and stabilisation conditions.

    \item Frequency noise coupling will also be mitigated by implementing optical isolation, improving mode matching between the LO and squeezed beams, and applying thermal stabilisation to critical components such as the nonlinear crystal and coupling optics. These steps aim to suppress slow drifts and mechanical noise that convert into phase fluctuations.

    \item Nonlinear gain in the PPLN waveguide will be characterised as a function of pump power and temperature. These measurements will support a detailed understanding of the amplification mechanism responsible for squeezing and allow predictions of how variations in gain influence the measured spectrum under phase noise.

    \item Squeezed light detection will be carried out using a balanced homodyne detector. The interference contrast (visibility) between the LO and signal will be optimised to ensure high fidelity measurement. Measurements of squeezing spectra will be compared with theoretical predictions incorporating frequency noise effects to validate the models and refine the noise budget.

    \item To further suppress residual phase noise, a coherent phase-locking system will be implemented using electro-optic phase modulation and active feedback. The performance of the phase lock will be assessed by measuring the residual phase noise and evaluating its impact on squeezing.

    \item In the final phase, all experimental parameters will be optimised to maximise observed squeezing. The level of squeezing will be quantified in decibels using spectral noise measurements and compared to theoretical expectations. Final documentation will summarise achieved performance, dominant noise sources, and system-level improvements needed to scale this approach to LIGO-class interferometric detectors.
\end{itemize}


%\section{Timeline}
% Describe timeline of your plan, explaining what you would do every 2 weeks.
%\begin{longtable}{|c|p{11cm}|}
%\hline
%\textbf{Timeframe} & \textbf{Milestone/Task} \\
%\hline
%Week 1--2 & Familiarisation with WOPA setup. Extract parameters from the experimental design, model the squeezed and anti-squeezed quadrature variances. \\
%\hline
%Week 3--4 & Derive noise coupling transfer functions for frequency and phase noise. Simulate squeezing degradation due to phase noise. Calculate allowable RMS phase noise for $>\SI{6}{\decibel}$ squeezing.  \\
%\hline
%Week 5--6 &  Simulate the impact of laser frequency noise and determine necessary stabilisation specifications. Compare experimental noise measurements to the theoretical noise budget and make improvmrnts.\\
%\hline
%Week 7--8 &  Finalise the noise budget document. Suggest design optimisations for phase/frequency noise suppression.\\
%\hline
%Week 9--10 &  Prepare final presentation/report and discuss implications for integration into a LIGO-like interferometer.   \\
%\hline
 \section{Timeline}
% Describe timeline of your plan, explaining what you would do every 2 weeks.
\begin{longtable}{|c|p{11cm}|}
\hline
\textbf{Timeframe} & \textbf{Milestone/Task} \\
\hline
Week 1--2 & Familiarisation with the WOPA experimental setup. Extract relevant optical and electronic parameters from the design. Model the variances of the squeezed and anti-squeezed quadratures using quantum optics formalism. Characterise the laser frequency noise spectrum using a frequency discriminator and spectrum analyser. Identify dominant noise contributions from the laser cavity, current source, and external optics. Implement initial modifications, such as optical isolation, improved mode matching, and thermal stabilisation—to mitigate frequency noise coupling into the squeezed quadrature. \\
\hline
Week 3--4 & Derive and simulate the noise coupling transfer functions that quantify how frequency and phase noise couple into the measured squeezing spectrum. Use these models to simulate squeezing degradation due to phase noise. Calculate tolerable RMS phase noise levels to ensure observable squeezing exceeds \SI{6}{\decibel}. Initiate measurements of nonlinear gain in the PPLN waveguide by varying pump power and temperature conditions. \\
\hline
Week 5--6 & Continue nonlinear gain characterisation and begin the first iteration of squeezed light detection using a balanced homodyne detector (BHD). Assess optical visibility and interference contrast between the local oscillator (LO) and squeezed signal field. Simulate and quantify the impact of laser frequency noise on the squeezing spectrum. Compare theoretical predictions with initial experimental measurements and identify stabilisation requirements. \\
\hline
Week 7--8 & Implement a coherent phase-locking system between the LO and signal beam using an electro-optic phase modulation scheme with active feedback. Characterise the phase lock's stability and residual noise. Finalise the noise budget by quantifying all significant degradation sources, including polarisation mismatch, gain imbalance in the BHD, mode-matching inefficiencies, and residual intensity noise. Suggest system-level improvements. \\
\hline
Week 9--10 & Optimise all experimental parameters to maximise the observed squeezing level. Evaluate squeezing in dB via spectral noise measurements and compare to theoretical predictions. Quantify the final level of quantum noise suppression achieved and prepare the final report. \\
\hline
\end{longtable}

%\end{longtable}


\begin{thebibliography}{9}

\bibitem{zetie_mzi}
K. P. Zetie, S. F. Adams, and R. M. Tocknell, 
\emph{How does a Mach–Zehnder interferometer work?}, 
Physics Education \textbf{35}(1), 46–48 (2000).

\bibitem{mckenzie}
K. McKenzie, 
\emph{Squeezing in the Audio Gravitational Wave Detection Band}, 
Ph.D. thesis, The Australian National University (2008).

\bibitem{goda}
K. Goda, 
\emph{Development of Techniques for Quantum-Enhanced Laser-Interferometric Gravitational-Wave Detectors}, 
Ph.D. thesis, Massachusetts Institute of Technology (2007).

\bibitem{saleh_teich}
M. C. Teich and B. E. A. Saleh, 
\emph{Squeezed States of Light}, 
Quantum Optics: Journal of the European Physical Society B \textbf{1}, 153–191 (1989); reprinted in \emph{Tutorials in Optics}, ed. D. T. Moore (Optical Society of America, 1992), ch. 3, pp. 29–52.

\bibitem{ligo_website}
LIGO Laboratory, \emph{LIGO Caltech Website}, \url{https://www.ligo.caltech.edu/}

\end{thebibliography}

\end{document}
