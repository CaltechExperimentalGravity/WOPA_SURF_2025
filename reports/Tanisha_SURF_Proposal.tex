\documentclass[colorlinks=true,pdfstartview=FitV,linkcolor=blue,
citecolor=red,urlcolor=magenta]{ligodoc}

\usepackage{siunitx}
\usepackage{graphicx}
\usepackage{amssymb}
\usepackage{amsmath}
\usepackage{longtable}
\usepackage{rotating}
\usepackage[usenames,dvipsnames]{color}
\usepackage{fancyhdr}
\usepackage{subfigure}
\usepackage{hyperref}

\ligodccnumber{T}{22}{xxxxx}{}{}
\title{}
\author{}
\begin{document}

\tableofcontents

\newpage
\section{Introduction}
% Introduce LIGO and bring in how your project contributes to LIGO research.
The Laser Interferometer Gravitational-Wave Observatory (LIGO) detects gravitational waves by measuring incredibly small distortions in spacetime using highly sensitive laser interferometers. One critical limitation to LIGO’s sensitivity, particularly at low frequencies, is quantum noise, which includes shot noise and radiation pressure noise. Squeezed light injection — a technique that reduces quantum noise — is thus vital to LIGO's performance enhancements.

This project directly contributes to this effort by developing a detailed noise budget analysis for a Waveguided Optical Parametric Amplification (WOPA) experiment. In this experiment, several decibels of quantum squeezing at \SI{1064}{\nano\meter} are generated by pumping a Periodically Poled Lithium Niobate (PPLN) waveguide with \SI{532}{\nano\meter} light. The squeezed \SI{1064}{\nano\meter} beam is then combined with a local oscillator (LO) to prepare it for eventual injection into an interferometer similar to those used in LIGO. Calculating the noise budget for frequency noise and phase noise in this system is essential to ensuring that we meet the squeezing levels ($> 6 dB$) required for meaningful improvements in LIGO sensitivity.

\section{Objective}
% Set goals for yourself here.
The main goals of this project are:

\begin{itemize}
    \item To calculate and quantify the noise budget, particularly focusing on frequency noise and phase noise contributions, for the WOPA setup described.
    \item To establish tolerances for phase stability that ensure more than \SI{6}{\decibel} of observed squeezing.
    \item To provide practical recommendations to mitigate these noise sources for successful integration of the squeezed source into an interferometric gravitational wave detector like LIGO.
\end{itemize}


\section{Approach}
% Describe a working plan. Cite an article like \cite{PDH}.
The project will be conducted in the following manner:

\begin{itemize}
    \item Model the squeezed light source using the theoretical framework of quantum optics, particularly the impact of frequency noise and phase noise on the squeezed quadratures.
    \item Analyze the WOPA optical setup, including the PPLN waveguide, LO phase shifter, and detection optics, using parameters extracted from the experimental layout.
    \item Develop noise transfer functions that describe how laser frequency noise and LO phase noise couple into the measured squeezed quadrature.
    \item Simulate the noise degradation using the mathematical relationship between squeezing, anti-squeezing, and phase noise.
    \item Validate noise budgets experimentally against squeezing degradation thresholds, setting allowable noise levels to meet $>\SI{6}{\decibel}$ observed squeezing.
    \item Propose methods to control phase noise, such as active feedback stabilization and passive noise reduction.
\end{itemize}

\section{Timeline}
% Describe timeline of your plan, explaining what you would do every 2 weeks.
\begin{longtable}{|c|p{11cm}|}
\hline
\textbf{Timeframe} & \textbf{Milestone/Task} \\
\hline
Week 1--2 & Familiarisation with WOPA setup. Extract parameters from the experimental design, model the squeezed and anti-squeezed quadrature variances. \\
\hline
Week 3--4 & Derive noise coupling transfer functions for frequency and phase noise. Simulate squeezing degradation due to phase noise. Calculate allowable RMS phase noise for $>\SI{6}{\decibel}$ squeezing.  \\
\hline
Week 5--6 &  Simulate the impact of laser frequency noise and determine necessary stabilisation specifications. Compare experimental noise measurements to the theoretical noise budget and make improvmrnts.\\
\hline
Week 7--8 &  Finalise the noise budget document. Suggest design optimisations for phase/frequency noise suppression.\\
\hline
Week 9--10 &  Prepare final presentation/report and discuss implications for integration into a LIGO-like interferometer.   \\
\hline
\end{longtable}


\begin{thebibliography}{9}
\bibitem{PDH} Authors. \emph{Paper Title}. Journal Abbreviation, Vol, Page numbers (YYYY).
\bibitem{controls} Authors. \emph{Website Title}. \url{link}.

\end{thebibliography}
\end{document}
