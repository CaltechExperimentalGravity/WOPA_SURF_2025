\documentclass[colorlinks=true,pdfstartview=FitV,linkcolor=blue,
citecolor=red,urlcolor=magenta]{ligodoc}

\usepackage{siunitx}
\usepackage{graphicx}
\usepackage{amssymb}
\usepackage{amsmath}
\usepackage{longtable}
\usepackage{rotating}
\usepackage[usenames,dvipsnames]{color}
\usepackage{fancyhdr}
\usepackage{subfigure}
\usepackage{hyperref}
\usepackage{float}
\ligodccnumber{T}{22}{xxxxx}{}{}
\title{\large{ Noise Characterisation and Suppression in a Waveguided Optical Parametric Amplification (WOPA) -Based Quantum Squeezing Experiment}}
\author{Tanisha Ray}
\begin{document}

\tableofcontents


\newpage

\section{Abstract}
Gravitational wave detectors such as Advanced LIGO rely on exquisite sensitivity to detect minuscule spacetime perturbations, making them susceptible to a variety of noise sources. Among the most fundamental is quantum noise, which arises from vacuum fluctuations entering the interferometer’s antisymmetric port. To mitigate this limitation, squeezed vacuum states—quantum states with reduced uncertainty in one quadrature—are injected to suppress shot noise. Currently, squeezing at LIGO is achieved via optical parametric oscillators (OPOs) using resonant cavities. This project explores Waveguided Optical Parametric Amplification (WOPA) as a cavity-free alternative that offers architectural simplicity and potential robustness against alignment instabilities. WOPA utilizes a single-pass configuration through a periodically poled lithium niobate (PPLN) waveguide, pumped with 532 nm light, to generate broadband squeezed vacuum states. \\
This work aims to develop a detailed noise and loss budget for the WOPA-based squeezing source. The analysis includes contributions from laser frequency and intensity noise, phase noise, polarization mismatch, propagation and coupling losses in the waveguide, imperfect mode matching, and gain imbalance at the balanced homodyne detector (BHD). The ultimate goal is to identify and limit these noise sources, thereby increasing the achievable level of squeezing. 
% This project serves as a precursor toward developing simplified, robust squeezing sources that may be suitable for future deployment in the gravitational detectors.
\section{Introduction}
\begin{figure}[H]
    \centering
    \includegraphics[height=15cm]{figures/Noise.png}
   \caption{Noise level of the various sources in Advanced LIGO \cite{mckenzie}}
    \label{fig:enter-label}
\end{figure}
Gravitational wave observatories like LIGO achieve remarkable strain sensitivities, detecting distortions in spacetime as small as $10^{-22}m/m$. This sensitivity is fundamentally constrained by quantum noise, which limits both the low-frequency and high-frequency performance of the interferometer. To suppress the high-frequency component—shot noise—LIGO currently injects squeezed vacuum states generated by cavity-based optical parametric oscillators (OPOs).
Current implementations of squeezing using optical cavities can achieve up to 4 dB of quantum noise suppression but are complex, alignment-sensitive, and require active stabilization\cite{goda}. These complexities pose limitations for scalability and robustness in long-term observatory operation.

WOPA presents an alternative approach by eliminating the need for resonant cavities. By using a periodically poled lithium niobate (PPLN) waveguide in a single-pass geometry, it simplifies the optical layout and offers a potentially more stable platform for squeezing generation.\\
 To explore more robust and simplified architectures, this project investigates Waveguided Optical Parametric Amplification (WOPA) in a table-top experimental setup. In this setup, several decibels of quantum squeezing at 1064 nm will be generated by pumping a Periodically Poled Lithium Niobate (PPLN)
waveguide with 532 nm light. The squeezed 1064 nm beam is then combined
with a local oscillator (LO) to prepare it for it's injection into an Mach–Zehnder-like balanced homodyne detector to measure the squeezing. The aim is to develop a detailed noise and loss budget for this WOPA-based squeezing source, including contributions from laser noise, propagation and coupling losses with the waveguide, and imperfections in detection and mode matching. By identifying and minimizing dominant noises, this work aims to improve the measured level of squeezing.\\
Thus, this project serves as a precursor toward developing simplified, robust squeezing sources that may be suitable for future deployment in the gravitational detectors. Additionally, the potential for extending WOPA-generated squeezing to applications in quantum-enhanced metrology, quantum neural networks, continuous-variable quantum computing, and coherent feedback control can be explored in future.

\section{Motivation}


Though WOPA presents an alternative approach of generating squeezed states by eliminating the need for resonant cavities, the performance of such systems is currently limited by a range of noise sources and optical losses. Our group’s initial implementation has achieved modest levels of squeezing ($\sim$0.2 dB) which is far behind the currently achievable squeezing level of it's alternative configuration OPO ($\sim 4 dB$). Our setup is currently constrained by coupling inefficiencies, path length mismatches, mode-matching errors, and the absence of active quantum noise locking.

My project is motivated by the need to rigorously quantify and address these limiting factors. By constructing a detailed noise and loss budget, I aim to identify and implement optimization strategies that will increase squeezing levels in the WOPA system. 
\section{Overview}
\begin{figure}[H]
    \centering
    \includegraphics[height=15cm]{figures/Setup.png}
   \caption{ Current WOPA experimental setup }
    \label{fig:enter-label}
\end{figure}

In this section the principal concepts used in this project will be addressed-
\subsection{Quantum Noise and Squeezed State Generation}
Quantum noise plays a central role in limiting the sensitivity of gravitational wave detectors across much of their detection bandwidth\cite{mckenzie}. This noise originates from vacuum fluctuations entering the interferometer’s antisymmetric port and comprises two principal components: \textbf{shot noise} and \textbf{radiation pressure noise}. Shot noise arises from fluctuations in the number of photons arriving at the photodetector, leading to uncertainty in power measurements. Radiation pressure noise, on the other hand, is due to momentum fluctuations of photons reflecting off the suspended test masses, causing perturbations in their positions\cite{mckenzie}.

These noise mechanisms are fundamentally rooted in the Heisenberg uncertainty principle, reflecting the non-commuting nature of position and momentum observables. Together, they define the Standard Quantum Limit (SQL), which sets the ultimate sensitivity bound of interferometric measurements in the absence of quantum non-demolition techniques. For a Michelson interferometer, the SQL for strain sensitivity is given by:
\begin{equation}
    h_{\mathrm{SQL}} = \sqrt{\frac{4\hbar}{m \Omega^2 L^2}}
\end{equation}
To surpass this limit, one can employ non-classical states of light known as squeezed states, in which quantum noise is redistributed such that uncertainty is reduced in one observable (quadrature) at the expense of the other. The electric field of the optical mode can be decomposed into amplitude ($X_1$) and phase ($X_2$) quadratures which relate to the bosonic creation ($a^\dagger$) and anhilation($a$) operators via:
\begin{equation}
    a = \frac{X_1 + i X_2}{2}, \quad a^\dagger = \frac{X_1 - i X_2}{2}
\end{equation}

Due to their non-commuting nature, these quadratures obey the uncertainty relation:
\begin{equation}
    \Delta X_1 \Delta X_2 \geq 1
\end{equation}
In vacuum and coherent states, both quadratures exhibit equal uncertainty ($\Delta X_1= \Delta X_2=1$). Squeezed states, however, achieve reduced uncertainty in one quadrature ($<1$) while increasing it in the conjugate quadrature ($>1$), thereby preserving the total uncertainty.\cite{saleh_teich}.\\
\begin{figure}[H]
    \centering
    \includegraphics[height=11cm]{figures/Wigner_functions.png}
   \caption{  Ball and stick picture for four states of light: (a) The coherent state, (b) The vacuum state, (c)
The amplitude squeezed states, and (d) The classically noisy state\cite{mckenzie} }
    \label{fig:enter-label}
\end{figure}



Squeezing is typically generated via nonlinear optical interactions in specialized crystals. When an electric field $E$ is applied to a nonlinear medium, the resulting polarization $P$ of the medium can be expressed as a power series expansion:
\begin{equation}
    P = \varepsilon_0 \left( \chi^{(1)} E + \chi^{(2)} E^2 + \chi^{(3)} E^3 + \cdots \right)
\end{equation}

Here, $\varepsilon_0$ is the vacuum permittivity and $\chi^{(n)}$ is the $n^{\text{th}}$-order nonlinear susceptibility.\\

In our experiment, a Periodically Poled Lithium Niobate (PPLN) waveguide exhibiting $\chi^{(2)}$ nonlinearity is used to generate squeezed states via a three-wave mixing processes of \textbf{degenerate Spontaneous Parametric Down-Conversion (SPDC)} which shall be explained next.\\

The $\chi^{(2)}$ nonlinear interactions can be classified as:
\begin{itemize}
    \item \textbf{Up-conversion:} $\omega_1 + \omega_2 = \omega_3$
    \item \textbf{Down-conversion:} $\omega_3 = \omega_1 + \omega_2$
\end{itemize}
When $\omega_1= \omega_2$ in an up-conversion process, it is called \textbf{Second Harmonic Generation (SHG)} and in a down conversion process, it is called\textbf{ degenerate} \textbf{Spontaneous Parametric Down-Conversion (SPDC)}. In the process of degenerate SPDC, when an intense and coherent pump field ($\omega_1$) is treated as undepleted and classical, the quantum interaction Hamiltonian takes the form:
\begin{equation}
\hat{H}_\text{int} = i \hbar \kappa \left( \hat{a}^{\dagger 2} e^{-i\theta} - \hat{a}^2 e^{i\theta} \right)
\end{equation}
where $\kappa$ depends on the nonlinear coefficient and the amplitude of the pump field. This Hamiltonian corresponds to the \textit{single-mode squeezing operator} acting on the vacuum state:
\begin{equation}
|\psi\rangle = \hat{S}(r) |0\rangle
\end{equation}
with
\begin{equation}
\hat{S}(r) = \exp\left[ \frac{1}{2} \left( r e^{-i\theta} \hat{a}^2 - r e^{i\theta} \hat{a}^{\dagger 2} \right) \right]
\end{equation}
Here, $r$ is the squeezing parameter proportional to the pump intensity and the effective nonlinearity, and $\theta$ defines the orientation of the squeezed quadrature in phase space.\\ 

Thus this process produces a \textbf{squeezed vacuum state} which is defined as a squeezed state with no coherent amplitude ($\langle a \rangle = 0$). \\
 
 \textbf{Optical Parametric Amplification (OPA)} is the stimulated counterpart of the SPDC process. In OPA, along with the pump field at frequency $\omega_p$, a coherent signal field (seed) at frequency $\omega_s$ is injected into the nonlinear medium. Through the second-order ($\chi^{(2)}$) nonlinearity of the medium, the signal field undergoes phase-sensitive amplification, while an idler field at frequency $\omega_i = \omega_p - \omega_s$ is simultaneously generated to conserve energy. The presence of the seed signal field fundamentally alters the output state: the amplification process is now dependent on the phase relationship between the signal and the pump.

Unlike SPDC, which results in a squeezed vacuum, OPA produces a \textit{squeezed coherent state} when a seed coherent signal is injected. In the special case where no input signal is injected ($\hat{a}_\text{in} = 0$), the OPA process reduces to SPDC and generates a squeezed vacuum state. Thus, OPA serves as a more general framework that encompasses both spontaneous and stimulated parametric processes. OPA is particularly valuable in practical implementations of squeezing in LIGO detectors where controlling the phase of the squeezed state is crucial.
\begin{figure}[H]
    \centering
    \includegraphics[height=10cm]{figures/OPA.png}
   \caption{  Schematic of OPA \cite{mckenzie}}
    \label{fig:enter-label}
\end{figure}


In our WOPA setup, the squeezed vacuum  state generated via SPDC  will then be detected using balanced homodyne detection.  
\subsection{Balanced Homodyne Detection}
The squeezed states generated via SPDC are typically analyzed using Balanced Homodyne Detection (BHD)—the standard method for quadrature measurements of optical fields [3]. The schematic of the setup is shown below.
\begin{figure}[H]
    \centering
    \includegraphics[height=11cm]{figures/BHD.png}
   \caption{  A balanced homodyne detector composed of a 50/50 beamsplitter, a coherent local
oscillator (LO) field, and a pair of photodiodes or a balanced photodetector. The input field is a
squeezed state of light or vacuum\cite{goda}}
    \label{fig:enter-label}
\end{figure}

In BHD, the signal field is combined with a strong reference beam known as the local oscillator (LO) on a symmetric beam splitter. The LO’s phase is tuned using a piezoelectric transducer to select the measurement quadrature. The output beams from the splitter are directed to two matched photodiodes (matched via gain balancing), and the resulting photocurrents are electronically subtracted to isolate the quadrature-dependent signal. This differential signal provides direct access to the squeezed quadrature, allowing us to make precise noise measurements in the  time domain. \\

However, the efficacy of BHD is limited by practical imperfections, such as gain imbalance between the photodiodes, path length mismatch between signal and LO arms, and residual technical noises such as laser intensity and phase fluctuations. As such, careful analysis of the BHD readout is essential to accurately quantify the level of squeezing.


\subsection{Phase Noise Characterization}
In a homodyne detection setup such as the WOPA configuration, the signal beam (carrying squeezed vacuum) and the local oscillator (LO) beam traverse distinct arms of a Mach-Zehnder-type interferometer before recombining at a beamsplitter. When these two arms are not precisely matched in length, the setup becomes susceptible to laser frequency noise, which introduces a time-dependent phase fluctuation into the readout and thereby degrades the observable squeezing.

Let the electric fields of the local oscillator and the squeezed signal be given by:
\begin{align}
    E_{\text{LO}}(t) &= E_0 \, e^{-i[\omega t + \phi_{\text{LO}}(t)]} \\
    E_{\text{sig}}(t) &= E_0' \, e^{-i[\omega t + \phi_{\text{sig}}(t)]}
\end{align}

Let the mismatch in optical path lengths between the LO and signal arms be \( \Delta L \). The phase accumulated by each beam over its respective path length depends on the instantaneous optical frequency \( \nu(t) = \nu_0 + \delta \nu(t) \), where \( \delta \nu(t) \) is the laser frequency noise. The corresponding wavenumber is:
\begin{equation}
    k(t) = \frac{2\pi \nu(t)}{c} = \frac{2\pi}{c} (\nu_0 + \delta \nu(t))
\end{equation}

Thus, the time-dependent phase acquired along a path of length \( L \) is:
\begin{equation}
    \phi(t) = k(t) L = \frac{2\pi}{c} \nu(t) L = \underbrace{\frac{2\pi \nu_0 L}{c}}_{\text{static}} + \underbrace{\frac{2\pi \delta \nu(t) L}{c}}_{\text{dynamic}}
\end{equation}

The phase difference between the two arms is then:
\begin{equation}
    \delta \phi(t) = \phi_{\text{LO}}(t) - \phi_{\text{sig}}(t) = \frac{2\pi \nu_0 \Delta L}{c} + \frac{2\pi \delta \nu(t) \Delta L}{c}
\end{equation}

Here, the first term is a static offset, which can be compensated via phase control. The second term is a fluctuating phase shift due to laser frequency noise:
\begin{equation}
    \delta \phi(t) \approx \frac{2\pi \Delta L}{c} \delta \nu(t)
\end{equation}

In our system, the laser exhibits approximately 75~MHz of frequency noise, i.e., \( \delta \nu \approx 75 \times 10^6 \)~Hz. This phase noise rotates the measurement quadrature and mixes the anti-squeezed component into the detected squeezed quadrature, degrading the observed squeezing level.

The measured squeezing level \( S_{\text{meas}} \) (in dB) in the presence of small phase noise \( \delta \phi \) is approximately:
\begin{equation}
    S_{\text{meas}} \approx -10 \log_{10} \left( e^{-2r} \cos^2 \delta \phi + e^{2r} \sin^2 \delta \phi \right)
\end{equation}
where \( r \) is the squeezing parameter, and the ideal squeezing level (in dB) is:
\begin{equation}
    S_{\text{ideal}} = -10 \log_{10}(e^{-2r})
\end{equation}

To observe more than 6~dB of squeezing (corresponding to \( r \approx 0.69 \), \( e^{2r} \approx 4 \), and \( e^{-2r} \approx 0.25 \)), the phase fluctuation must satisfy \( \delta \phi \lesssim 0.14 \)~rad to limit degradation to under 1~dB. Substituting into the phase noise expression:
\begin{align}
    \Delta L &\lesssim \frac{\delta \phi_{\text{max}} \cdot c}{2\pi \, \delta \nu} \\
             &\lesssim \frac{0.14 \cdot 3 \times 10^8 \, \text{m/s}}{2\pi \cdot 75 \times 10^6 \, \text{Hz}} \\
             &\lesssim 0.089 \, \text{m} \approx 8.9 \, \text{cm}
\end{align}

To observe more than 6~dB of squeezing in the presence of 75~MHz laser frequency noise, the optical path length difference between the LO and signal arms must be maintained below approximately \textbf{9~cm}. Minimising this mismatch is essential to suppress phase noise and preserve the integrity of the squeezed state at the readout.

\section{Approach}

Our goal was to generate and characterize squeezed vacuum states using a waveguide-based Optical Parametric Amplification (WOPA) setup and ultimately maximize the squeezing level achieved by optimizing the existing setup. Toward this, I carried out a series of experiments that span optical alignment, polarization characterization, shot noise calibration, squeezing measurement, and PZT-based phase modulation.

\subsection{System Characterization and Setup Calibration}

To establish the reliability of the experimental setup, we first characterized the polarization properties, nonlinear gain of the crystal, mode-matching between the local oscillator (LO) and signal paths, and gain balancing in the balanced homodyne detection (BHD) system using the Moku:Lab and Moku:Go.

\subsubsection{Polarization Measurements}

By intentionally misaligning the output fiber from the crystal and inserting a rotatable polarizing beam cube (PBC), we measured the polarization of the beams:
\begin{itemize}
    \item 1064\,nm beam: $\sim$81\% linear polarization
    \item 532\,nm beam: $\sim$93\% linear polarization
\end{itemize}
The incomplete extinction during rotation of the PBC indicated elliptical polarization, likely due to fiber misalignment or mechanical stress. Additionally, we measured:
\begin{itemize}
    \item 36\% coupling efficiency for 1064\,nm into the bare fiber
    \item 47\% coupling efficiency for 532\,nm into the bare fiber
\end{itemize}

\subsubsection{Nonlinear Gain of PPLN Crystal}

We aligned a 1064\,nm beam through the PPLN waveguide under unphased conditions (room temperature), and then tuned the crystal to 52.5$^\circ$C for phase matching to maximize second harmonic generation (SHG). Using a dichroic mirror to isolate 532\,nm light, we measured the the normalized conversion efficiency of the crystal using the formula:
\begin{figure}[H]
    \centering
    \includegraphics[height=5cm]{figures/NLG.png}
   \caption{  PPLN Waveguide- SHG \cite{manual}}
    \label{fig:enter-label}
\end{figure}
\[
\text{Normalize Conversion Efficiency}~\left(\frac{\%}{\text{Watt}/\text{cm}^2}\right) = 
\frac{
    \dfrac{X_{\text{SHG}}~[\mu\text{W}]}{(1 - R_{\text{OUTPUT}})(1 - L_{\text{LOSS}})}
}{
    \left( \dfrac{Y_{\text{PUMP}}~[\text{mW}]}{(1 - R_{\text{OUTPUT}})(1 - L_{\text{LOSS}})} \right)^2 \times (L~[\text{cm}])^2
}
\times 100\%
\]

\textbf{Result:} With a pump power of 9.5 mW and SHG power of 123 µW, the efficiency was found to be $\sim$60\%/W.

\subsubsection{Mode Matching and Visibility}

Using a beam profiler, we matched the mode profiles of the LO and signal paths to within $\sim$10--20\,$\mu$m. The visibility, determined from the DC photodetector output during alignment using:

\[
V = \frac{I_{\text{max}} - I_{\text{min}}}{I_{\text{max}} + I_{\text{min}}}
\]
\textbf{Result:} Achieved BHD visibility was $V \approx 0.6$.

\subsubsection{Gain Balancing and Shot Noise Measurement}

We directed $\sim$0.6\,mW of 1064\,nm LO power to each PD, while blocking the signal path. The Moku:Lab's FIR filter builder was configured to subtract the PD outputs with:
\[
\text{North PD gain: } +1.2 \quad \text{South PD gain: } -0.9
\]
A flat trace at higher frequencies confirmed shot-noise-limited detection after gain balancing the two PDs, with values matching theoretical predictions. The theoretical prediction was made using the formula:
\[
S = \left( \sqrt{2 h \nu P} \right)(QR)(TRI)(R)(\text{Gain}) \approx 1.33 \times 10^{-6}~\frac{V}{\sqrt{\text{Hz}}}
\]
\begin{figure}[H]
    \centering
    \includegraphics[height=11cm]{figures/BHD_gainb.png}
   \caption{ Shot Noise- Gain balanced state }
    \label{fig:enter-label}
\end{figure}
The differential signal was then band-passed between 2.5--7\,MHz, amplified by 5\,dB, and routed to Moku:Go for squeezing measurement.

\subsection{Squeezing Measurement}
To measure squeezing, we first recorded the unsqueezed variance by blocking the signal arm and detecting the local oscillator beam alone. This yielded a baseline quadrature variance of approximately 50~mV. Upon enabling the 532~nm pump beam at around 2~mW, quadrature noise modulation was observed due to spontaneous phase drift between the squeezed signal and the local oscillator.

The level of squeezing and anti-squeezing in decibels was calculated using:

\[
\text{Level}_{\text{dB}} = 20 \log_{10} \left( \frac{V_{\text{squeezed/anti-squeezed}}}{V_{\text{vacuum}}} \right)
\]

\textbf{Result:} Using this relation, the measured squeezing was approximately $-0.14$~dB and anti-squeezing was $+0.29$~dB, indicating successful generation of squeezed states.\\


Next we varied the pump power and made the squeezing measurements. The plot obtained was as follows:
\begin{figure}[H]
    \centering
    \includegraphics[height=11cm]{figures/Squeezing_diff_pumpp.png}
   \caption{ Squeezing level at  different pump powers}
    \label{fig:enter-label}
\end{figure}


\subsection{PZT-Based Phase Modulation}

To enable dynamic phase scanning, we used a Piezo Transducer (PZT)- mounted mirror in the signal path. A 100\,Hz triangular waveform (0--10\,V) was amplified by $\times15$ and fed into the PZT. The same waveform was used as a reference in Moku:Go. Phase modulation was verified through the interference pattern at the BHD output, showing:
\begin{itemize}
    \item Slightly $>$1 wavelength sweep for 75\,V input
    \item $\sim$2 wavelength sweep for 150\,V input
\end{itemize}
\begin{figure}[H]
    \centering
    \includegraphics[height=11cm]{figures/PZT_75_labelled.png}
   \caption{  PZT sweep at 75V}
    \label{fig:enter-label}
\end{figure}
\begin{figure}[H]
    \centering
    \includegraphics[height=11cm]{figures/PZT_150_labelled.png}
   \caption{  PZT sweep at 150V}
    \label{fig:enter-label}
\end{figure}
This enables observation of two squeezing and anti-squeezing cycles each, in future measurements within one sweep of the PZT at 75V.

\subsection{PZT Transfer Function and Noise Locking}

To measure the transfer function of the PZT to characterize any mechanical resonances in the system we designed a feedback loop.

The loop diagram of our system is as follows:
\begin{figure}[H]
    \centering
    \includegraphics[height=6cm]{figures/PZT_loop.png}
   \caption{Feedback loop for drift noise locking}
    \label{fig:enter-label}
\end{figure}
\begin{itemize}
    \item The error signal (DC PD difference) was fed into an integrator (Moku PID module)
    \item Controller output was amplified using SRS-560 and a PZT driver (10\,V/V and 15\,V/V)
\end{itemize}
\begin{figure}[H]
    \centering
    \includegraphics[height=11cm]{figures/MOKU_PID_Phase_Locking.png}
   \caption{Phase drift locked state}
    \label{fig:enter-label}
\end{figure}

\begin{figure}[H]
    \centering
    \includegraphics[height=11cm]{figures/Locked_Unlocked_State.png}
   \caption{Transtion from locked to unlocked state on turning of the controller}
    \label{fig:enter-label}
\end{figure}
Efforts to lock using Moku alone failed due to voltage limitations. However, with SRS-560 to amplify the Moku Pro output, we achieved noise locking and demonstrated the transition from locked to unlocked states on turning off the controller. This paves the way for locked-phase squeezing measurements with active modulation in future.

\subsection{Noise ASD Characterization}

We characterized the noise amplitude spectral density (ASD) of both Moku:Pro and the PDs:
\begin{itemize}
    \item Moku:Pro input noise was measured across segmented frequency ranges (100\,Hz to 300\,MHz) and stitched together
    \item PD dark noise was measured with all light sources off
\end{itemize}
\begin{figure}[H]
    \centering
    \includegraphics[height=11cm]{figures/Noise_ASD_Moku_Pds.png}
   \caption{ASD Noise Spectrum}
    \label{fig:enter-label}
\end{figure}
Both spectra qualitatively matched those in their respective manuals, with minor amplitude deviations that will be investigated further. These measurements are essential for quantifying squeezing beyond the noise floor.


\section{Future Plan Of Action}

\begin{itemize}
    \item Understand the spatial and polarization modes supported by the PPLN crystal waveguide.
    \item Optimize the polarization entering the optical fiber to maximize second harmonic generation (SHG) and consequently maximize the squeezing levels.
    \item Maximize interference visibility by measuring the amplitude of the 1064 nm signal after the optical parametric amplifier (OPA) using intense 532nm pumd and a weak 1064 nm seed.
    \item Estimate optical losses using manufacturer specifications, then iteratively replace those with experimentally measured values to calculate the expected squeezing and required visibility.
    \item Adjust interferometer arm length precisely to minimize the phase noise. Investigate and mitigate the laser intensity noise through suitable techniques.
    \item Compare theoretical expectations with measured results to identify the limiting factors in squeezing performance and systematically optimize them. 
\end{itemize}


\begin{thebibliography}{9}

\bibitem{zetie_mzi}\label{1}
K. P. Zetie, S. F. Adams, and R. M. Tocknell, 
\emph{How does a Mach–Zehnder interferometer work?}, 
Physics Education \textbf{35}(1), 46–48 (2000).

\bibitem{mckenzie}
K. McKenzie, 
\emph{Squeezing in the Audio Gravitational Wave Detection Band}, 
Ph.D. thesis, The Australian National University (2008).

\bibitem{goda}
K. Goda, 
\emph{Development of Techniques for Quantum-Enhanced Laser-Interferometric Gravitational-Wave Detectors}, 
Ph.D. thesis, Massachusetts Institute of Technology (2007).

\bibitem{saleh_teich}
M. C. Teich and B. E. A. Saleh, 
\emph{Squeezed States of Light}, 
Quantum Optics: Journal of the European Physical Society B \textbf{1}, 153–191 (1989); reprinted in \emph{Tutorials in Optics}, ed. D. T. Moore (Optical Society of America, 1992), ch. 3, pp. 29–52.

\bibitem{ligo_website}
LIGO Laboratory, \emph{LIGO Caltech Website}, \url{https://www.ligo.caltech.edu/}
\bibitem{manual}
HC Photonics Corp.
PPLN Waveguide Manual
\end{thebibliography}
\end{document}
\end{document}