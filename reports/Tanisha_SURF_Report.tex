\documentclass[colorlinks=true,pdfstartview=FitV,linkcolor=blue,
citecolor=red,urlcolor=magenta]{ligodoc}

\usepackage{siunitx}
\usepackage{graphicx}
\usepackage{amssymb}
\usepackage{amsmath}
\usepackage{longtable}
\usepackage{rotating}
\usepackage[usenames,dvipsnames]{color}
\usepackage{fancyhdr}
\usepackage{subfigure}
\usepackage{hyperref}
\usepackage{float}

\usepackage{subcaption}
\ligodccnumber{T}{22}{xxxxx}{}{}
\title{\large{ Noise Characterisation and Suppression in a Waveguided Optical Parametric Amplification (WOPA) -Based Quantum Squeezing Experiment}}
\author{Tanisha Ray}
\begin{document}

\tableofcontents


\newpage

\section{Abstract}
Gravitational wave detectors such as Advanced LIGO rely on exquisite sensitivity to detect minuscule spacetime perturbations, making them susceptible to a variety of noise sources. Among the most fundamental is quantum noise, which arises from vacuum fluctuations entering the interferometer’s antisymmetric port. To mitigate this limitation, squeezed vacuum states—quantum states with reduced uncertainty in one quadrature—are injected to suppress shot noise. Currently, squeezing at LIGO is achieved via optical parametric oscillators (OPOs) using resonant cavities. This project explores Waveguided Optical Parametric Amplification (WOPA) as a cavity-free alternative that offers architectural simplicity and potential robustness against alignment instabilities. WOPA utilizes a single-pass configuration through a periodically poled lithium niobate (PPLN) waveguide, pumped with 532 nm light, to generate broadband squeezed vacuum states at 1064nm. \\
This work aims to develop a detailed noise and loss budget for the WOPA-based squeezing source. The analysis includes contributions from laser frequency and intensity noise, phase noise, polarization mismatch, propagation and coupling losses in the waveguide, imperfect mode matching, and gain imbalance at the balanced homodyne detector (BHD). The ultimate goal is to identify and limit these noise sources, thereby increasing the achievable level of squeezing. 
% This project serves as a precursor toward developing simplified, robust squeezing sources that may be suitable for future deployment in the gravitational detectors.
\section{Introduction}
\begin{figure}[H]
    \centering
    \includegraphics[height=15cm]{figures/Noise.png}
   \caption{Noise level of the various sources in Advanced LIGO \cite{mckenzie}}
    \label{fig:enter-label}
\end{figure}
\textbf{Gravitational waves} are perturbations or \textit{ripples} in the fabric of space-time, generated by some of the most violent and energetic astrophysical phenomena in the universe. Predicted by Albert Einstein in 1916 as a consequence of his general theory of relativity, gravitational waves arise from the acceleration of massive bodies—such as binary systems of neutron stars or black holes—which distort space-time in a dynamic fashion. These distortions propagate outward at the speed of light, carrying information about their astrophysical origins and providing profound insights into the fundamental nature of gravity.

\textbf{The Laser Interferometer Gravitational-Wave Observatory (LIGO)} is a pioneering facility designed to detect these elusive signals through the technique of laser interferometry. Each LIGO detector comprises two orthogonal arms, each 4 kilometers in length, forming a Michelson interferometer. A laser beam is split at the beam splitter and directed down both arms, where it is reflected off highly isolated mirrors (test masses) before recombination. A passing gravitational wave induces a differential change in arm lengths—stretching one while compressing the other—which leads to a shift in the interference pattern of the recombined light. Measurement of this interference enables the detection and characterization of gravitational wave signals.

LIGO achieves a remarkable strain sensitivity, capable of detecting spacetime distortions as small as 
\(10^{-22} \, \text{m/m}\). This extraordinary sensitivity is fundamentally constrained by \textbf{quantum noise}, which limits the detector's performance at both low and high frequencies. Specifically, the quantum fluctuations of the electromagnetic field give rise to amplitude and phase uncertainties at the photodetector readout, which manifest as shot noise at high frequencies and radiation pressure noise at low frequencies.

To mitigate this quantum noise, LIGO employs \textbf{squeezed vacuum states}, wherein the electromagnetic ground state is engineered to reduce the uncertainty in either the phase or amplitude quadrature at the expense of the other, consistent with the Heisenberg uncertainty principle. To tailor this squeezing across different frequencies, LIGO utilizes frequency-dependent squeezing, made possible through the implementation of 300-meter-long filter cavities at both observatory sites of Hanford and Livingston. These cavities reflect the squeezed vacuum states prior to their injection into the interferometer, allowing the quadrature angle of squeezing to rotate as a function of frequency.

Currently, squeezed vacuum states at LIGO are generated using cavity-based \textbf{Optical Parametric Oscillators (OPOs)}~\cite{goda}. These systems employ nonlinear crystals within optical resonators and have reported producing up to 4 dB (Hanford site) and 5.8 dB (Livingston site) of squeezing. However, these implementations are inherently complex, highly sensitive to optical alignment, and require active stabilization of the cavity length. Such challenges pose limitations in terms of robustness, scalability, and long-term operation of large-scale detectors.

To address these limitations, \textbf{Waveguide Optical Parametric Amplification (WOPA)} emerges as a promising alternative that eliminates the need for resonant cavities. By employing a periodically poled lithium niobate (PPLN) waveguide in a single-pass geometry, WOPA significantly simplifies the optical architecture while providing a potentially more stable and compact platform for squeezed light generation.

In this project, we explore the feasibility of WOPA-based squeezing through a tabletop experimental setup. Quantum squeezing at 1064 nm will be generated by pumping a PPLN waveguide with 532 nm light. The resulting squeezed vacuum at 1064 nm will be combined with a local oscillator in a Mach-Zehnder-like balanced homodyne detector, enabling direct measurement of the squeezing level. A comprehensive \textbf{noise and loss budget} will be developed, taking into account contributions from laser amplitude and phase noise, \textbf{coupling and propagation losses in the waveguide}, and inefficiencies in \textbf{detection and mode matching}. By identifying and minimizing dominant noise sources, this work aims to \textbf{enhance the observed squeezing level} and assess the viability of WOPA as a replacement for cavity-based systems.

This investigation serves as a critical step toward developing simplified, compact, and robust sources of squeezed light well-suited for future integration into gravitational-wave observatories. Furthermore, the underlying architecture offers promising pathways for broader applications in quantum-enhanced metrology, continuous-variable quantum information processing, quantum neural networks, and coherent feedback control.
\section{Motivation}


Though WOPA presents an alternative approach of generating squeezed states by eliminating the need for resonant cavities, the performance of such systems is currently limited by a range of noise sources and optical losses. Our group’s initial implementation has achieved modest levels of squeezing ($\sim$0.2 dB) which is far behind the currently achievable squeezing level of it's alternative configuration OPO ($\sim 4 dB$). Our setup is currently constrained by coupling inefficiencies, path length mismatches, mode-matching errors, and the absence of active quantum noise locking.

My project is motivated by the need to rigorously quantify and address these limiting factors. By constructing a detailed noise and loss budget, I aim to identify and implement optimization strategies that will increase squeezing levels in the WOPA system. 


\section{Overview}
In this section, the principal concepts used in this project will be addressed-
\subsection{Quantum Noise and Squeezed State Generation}
The sensitivity of Advanced LIGO is so great that even the discrete nature of light yields measurable effects. In fact, quantization imposes the most significant barrier to improving LIGO’s sensitivity across its entire measurement band. This quantum noise arises fundamentally from the interaction between quantum vacuum fluctuations and the high-power circulating laser field within the interferometer, entering primarily through the readout (dark) port. These vacuum fluctuations are described in terms of the amplitude and phase quadratures of the electromagnetic field, each contributing a distinct form of quantum noise.

Variance in the vacuum’s phase quadrature leads to shot noise, which is dominant at high frequencies. This noise manifests as fluctuations in photon arrival times at the photodetector, introducing statistical uncertainty in the measurement of small phase shifts caused by passing gravitational waves. At the other end of the spectrum, fluctuations in the amplitude quadrature give rise to radiation pressure noise, which becomes significant at low frequencies. Here, quantum fluctuations in the number of photons impinging on
Quantum noise plays a central role in limiting the sensitivity of gravitational wave detectors across their entire detection bandwidth \cite{mckenzie}. This noise originates from the interaction between quantum vacuum fluctuations and the high-power circulating laser field within the interferometer, entering primarily through the interferometer’s antisymmetric port. Quantum noise comprises two principal components: \textbf{shot noise} and \textbf{radiation pressure noise}. Shot noise arises from variance in the vacuum’s phase quadrature and manifests as fluctuations in the number of photons arriving at the photodetector, leading to uncertainty in power measurements predominantly at high frequencies. Radiation pressure noise, on the other hand, is due to fluctuations in the amplitude quadrature, manifesting as momentum fluctuations of photons reflecting off the suspended test masses, causing perturbations in their positions dominating at low frequencies. \cite{mckenzie}.

These noise mechanisms are fundamentally rooted in the Heisenberg uncertainty principle, reflecting the non-commuting nature of position and momentum observables. Together, they define the Standard Quantum Limit (SQL), which sets the ultimate sensitivity bound of interferometric measurements in the absence of quantum non-demolition techniques. For a Michelson interferometer, the SQL for strain sensitivity is given by:
\begin{equation}
    h_{\mathrm{SQL}} = \sqrt{\frac{4\hbar}{m \Omega^2 L^2}}
\end{equation}
To surpass this limit, one can employ non-classical states of light known as squeezed states, in which quantum noise is redistributed such that uncertainty is reduced in one observable (quadrature) at the expense of the other. The electric field of the optical mode can be decomposed into amplitude ($X_1$) and phase ($X_2$) quadratures which relate to the bosonic creation ($a^\dagger$) and anhilation($a$) operators via:
\begin{equation}
    a = \frac{X_1 + i X_2}{2}, \quad a^\dagger = \frac{X_1 - i X_2}{2}
\end{equation}

Due to their non-commuting nature, these quadratures obey the uncertainty relation:
\begin{equation}
    \Delta X_1 \Delta X_2 \geq 1
\end{equation}
In vacuum and coherent states, both quadratures exhibit equal uncertainty ($\Delta X_1= \Delta X_2=1$). Squeezed states, however, achieve reduced uncertainty in one quadrature ($<1$) while increasing it in the conjugate quadrature ($>1$), thereby preserving the total uncertainty.\cite{saleh_teich}.\\
\begin{figure}[H]
    \centering
    \includegraphics[height=11cm]{figures/Wigner_functions.png}
   \caption{  Ball and stick picture for four states of light: (a) The coherent state, (b) The vacuum state, (c)
The amplitude squeezed states, and (d) The classically noisy state\cite{mckenzie} }
    \label{fig:enter-label}
\end{figure}



Squeezing is typically generated via nonlinear optical interactions in specialized crystals. When an electric field $E$ is applied to a nonlinear medium, the resulting polarization $P$ of the medium can be expressed as a power series expansion:
\begin{equation}
    P = \varepsilon_0 \left( \chi^{(1)} E + \chi^{(2)} E^2 + \chi^{(3)} E^3 + \cdots \right)
\end{equation}

Here, $\varepsilon_0$ is the vacuum permittivity and $\chi^{(n)}$ is the $n^{\text{th}}$-order nonlinear susceptibility.\\

The $\chi^{(2)}$ nonlinear interactions can be classified as:
\begin{itemize}
    \item \textbf{Up-conversion:} $\omega_1 + \omega_2 = \omega_3$
    \item \textbf{Down-conversion:} $\omega_3 = \omega_1 + \omega_2$
\end{itemize}
When $\omega_1= \omega_2$ in an up-conversion process, it is called \textbf{Second Harmonic Generation (SHG)} and in a down-conversion process, it is called\textbf{ degenerate} \textbf{Spontaneous Parametric Down-Conversion (SPDC)}. In the process of degenerate SPDC, when an intense and coherent pump field ($\omega_1$) is treated as undepleted and classical, the quantum interaction Hamiltonian takes the form:
\begin{equation}
\hat{H}_\text{int} = i \hbar \kappa \left( \hat{a}^{\dagger 2} e^{-i\theta} - \hat{a}^2 e^{i\theta} \right)
\end{equation}
where $\kappa$ depends on the nonlinear coefficient and the amplitude of the pump field. This Hamiltonian corresponds to the \textit{single-mode squeezing operator} acting on the vacuum state:
\begin{equation}
|\psi\rangle = \hat{S}(r) |0\rangle
\end{equation}
with
\begin{equation}
\hat{S}(r) = \exp\left[ \frac{1}{2} \left( r e^{-i\theta} \hat{a}^2 - r e^{i\theta} \hat{a}^{\dagger 2} \right) \right]
\end{equation}
Here, $r$ is the squeezing parameter proportional to the pump intensity and the effective nonlinearity, and $\theta$ defines the orientation of the squeezed quadrature in phase space.\\ 

Thus, this process produces a \textbf{squeezed vacuum state} which is defined as a squeezed state with no coherent amplitude ($\langle a \rangle = 0$). \\
 
 \textbf{Optical Parametric Amplification (OPA)} is the stimulated counterpart of the SPDC process. In OPA, along with the pump field at frequency $\omega_p$, a coherent signal field (seed) at frequency $\omega_s$ is injected into the nonlinear medium. Through the second-order ($\chi^{(2)}$) nonlinearity of the medium, the signal field undergoes phase-sensitive amplification, while an idler field at frequency $\omega_i = \omega_p - \omega_s$ is simultaneously generated to conserve energy. The presence of the seed signal field fundamentally alters the output state: the amplification process is now dependent on the phase relationship between the signal and the pump.

Unlike SPDC, which results in a squeezed vacuum, OPA produces a \textit{squeezed coherent state} when a seed coherent signal is injected. In the special case where no input signal is injected ($\hat{a}_\text{in} = 0$), the OPA process reduces to SPDC and generates a squeezed vacuum state. Thus, OPA serves as a more general framework that encompasses both spontaneous and stimulated parametric processes.\\

\begin{figure}[H]
    \centering
    \includegraphics[height=10cm]{figures/OPA.png}
   \caption{  Schematic of OPA \cite{mckenzie}}
    \label{fig:enter-label}
\end{figure}

In our experiment, we inject a 532 nm beam as pump power into a Periodically Polled Lithium Niobate crystal (PPLN) to create vacuum-squeezed states of 1064 nm via Spontaneous Parametric Down Conversion (SPDC).  These vacuum-squeezed states are then detected using balanced homodyne detection.  
\subsection{Balanced Homodyne Detection}
The squeezed states generated via SPDC are typically analyzed using Balanced Homodyne Detection (BHD)—the standard method for quadrature measurements of optical fields \cite{goda}. The schematic of the setup is shown below.
\begin{figure}[H]
    \centering
    \includegraphics[height=11cm]{figures/BHD.png}
   \caption{  A balanced homodyne detector composed of a 50/50 beamsplitter, a coherent local
oscillator (LO) field, and a pair of photodiodes or a balanced photodetector. The input field is a
squeezed state of light or vacuum\cite{goda}}
    \label{fig:enter-label}
\end{figure}

In BHD, the signal field is combined with a strong reference beam known as the local oscillator (LO) on a symmetric beam splitter. The LO’s phase is tuned using a piezoelectric transducer to select the measurement quadrature. The output beams from the splitter are directed to two matched photodiodes (matched via gain balancing), and the resulting photocurrents are electronically subtracted to isolate the quadrature-dependent signal. This differential signal provides direct access to the squeezed quadrature, allowing us to make precise noise measurements in the  time domain. \\

However, the efficacy of BHD is limited by practical imperfections, such as gain imbalance between the photodiodes, path length mismatch between signal and LO arms, and residual technical noises such as laser intensity and phase fluctuations. As such, careful analysis of the BHD readout is essential to accurately quantify the level of squeezing.


\subsection{Phase Noise Characterization}
In a homodyne detection setup such as the WOPA configuration, the signal beam (carrying squeezed vacuum) and the local oscillator (LO) beam traverse distinct arms of a Mach-Zehnder-type interferometer before recombining at a beamsplitter. When these two arms are not precisely matched in length, the setup becomes susceptible to laser frequency noise, which introduces a time-dependent phase fluctuation into the readout and thereby degrades the observable squeezing.

Let the electric fields of the local oscillator and the squeezed signal be given by:
\begin{align}
    E_{\text{LO}}(t) &= E_0 \, e^{-i[\omega t + \phi_{\text{LO}}(t)]} \\
    E_{\text{sig}}(t) &= E_0' \, e^{-i[\omega t + \phi_{\text{sig}}(t)]}
\end{align}

Let the mismatch in optical path lengths between the LO and signal arms be \( \Delta L \). The phase accumulated by each beam over its respective path length depends on the instantaneous optical frequency \( \nu(t) = \nu_0 + \delta \nu(t) \), where \( \delta \nu(t) \) is the laser frequency noise. The corresponding wavenumber is:
\begin{equation}
    k(t) = \frac{2\pi \nu(t)}{c} = \frac{2\pi}{c} (\nu_0 + \delta \nu(t))
\end{equation}

Thus, the time-dependent phase acquired along a path of length \( L \) is:
\begin{equation}
    \phi(t) = k(t) L = \frac{2\pi}{c} \nu(t) L = \underbrace{\frac{2\pi \nu_0 L}{c}}_{\text{static}} + \underbrace{\frac{2\pi \delta \nu(t) L}{c}}_{\text{dynamic}}
\end{equation}

The phase difference between the two arms is then:
\begin{equation}
    \delta \phi(t) = \phi_{\text{LO}}(t) - \phi_{\text{sig}}(t) = \frac{2\pi \nu_0 \Delta L}{c} + \frac{2\pi \delta \nu(t) \Delta L}{c}
\end{equation}

Here, the first term is a static offset, which can be compensated via phase control. The second term is a fluctuating phase shift due to laser frequency noise:
\begin{equation}
    \delta \phi(t) \approx \frac{2\pi \Delta L}{c} \delta \nu(t)
\end{equation}

In our system, the laser exhibits approximately 75~MHz of frequency noise, i.e., \( \delta \nu \approx 75 \times 10^6 \)~Hz. This phase noise rotates the measurement quadrature and mixes the anti-squeezed component into the detected squeezed quadrature, degrading the observed squeezing level.

The measured squeezing level \( S_{\text{meas}} \) (in dB) in the presence of small phase noise \( \delta \phi \) is approximately:
\begin{equation}
    S_{\text{meas}} \approx -10 \log_{10} \left( e^{-2r} \cos^2 \delta \phi + e^{2r} \sin^2 \delta \phi \right)
\end{equation}
where \( r \) is the squeezing parameter, and the ideal squeezing level (in dB) is:
\begin{equation}
    S_{\text{ideal}} = -10 \log_{10}(e^{-2r})
\end{equation}

To observe more than 6~dB of squeezing (corresponding to \( r \approx 0.69 \), \( e^{2r} \approx 4 \), and \( e^{-2r} \approx 0.25 \)), the phase fluctuation must satisfy \( \delta \phi \lesssim 0.14 \)~rad to limit degradation to under 1~dB. Substituting into the phase noise expression:
\begin{align}
    \Delta L &\lesssim \frac{\delta \phi_{\text{max}} \cdot c}{2\pi \, \delta \nu} \\
             &\lesssim \frac{0.14 \cdot 3 \times 10^8 \, \text{m/s}}{2\pi \cdot 75 \times 10^6 \, \text{Hz}} \\
             &\lesssim 0.089 \, \text{m} \approx 8.9 \, \text{cm}
\end{align}

To observe more than 6~dB of squeezing in the presence of 75~MHz laser frequency noise, the optical path length difference between the LO and signal arms must be maintained below approximately \textbf{9~cm}. Minimising this mismatch is essential to suppress phase noise and preserve the integrity of the squeezed state at the readout.

\section{Approach}
\subsection{Experimental Setup}

\begin{figure}[H]
    \centering
    \includegraphics[height=15cm]{figures/Setup.png}
   \caption{ Current WOPA experimental setup }
    \label{fig:enter-label}
\end{figure}
\subsubsection{Squeezed State Generation}

To generate a squeezed vacuum state, a continuous-wave (CW) single-frequency laser source (Innolight Diabolo) provides both 1064\,nm and its frequency-doubled 532\,nm output. The 532\,nm beam is used for pumping the nonlinear waveguide. To ensure proper polarization and spatial mode quality, the beam is first passed through a half-wave plate (HWP) followed by a polarizing beam splitter (PBS). The HWP is adjusted to rotate the linear polarization axis such that only the desired polarization along the extraordinary axis of the nonlinear crystal is transmitted through the PBS. This is essential for efficient coupling into the nonlinear waveguide, as the nonlinear interaction strength in periodically poled lithium niobate (PPLN) is highly polarization-dependent.\\

The beam is then guided through a series of optical elements including steering mirrors and lenses. These optics serve to both direct the beam spatially and adjust its beam waist to match the fiber coupling requirements. Another HWP and PBS pair downstream further purifies the polarization and allows for fine-tuning of the pump power by rotating the polarization fraction incident on the PBS. The purified and mode-shaped 532\,nm beam is then tightly focused using a 40x microscope objective lens onto the input facet of an optical fiber, which is butt-coupled to the PPLN waveguide.\\

Within the PPLN waveguide, optical parametric amplification (OPA) occurs: the high-intensity 532\,nm pump interacts with vacuum fluctuations to produce a squeezed vacuum state at 1064\,nm in the degenerate signal mode via SPDC as described above. The waveguide’s tight confinement, quasi-phase-matching, and high nonlinear coefficient make it an excellent platform for generating strong squeezing.

\paragraph{Potential Limitations in Squeezed State Generation:}
\begin{enumerate}
    \item \textbf{Polarization Mismatch:} Squeezing is quadrature-selective—amplitude and phase quadratures are orthogonal, and the nonlinear gain is maximized only when the pump polarization aligns with the nonlinear tensor axis of the PPLN. Any misalignment can lead to suboptimal interaction or mixing of quadratures, reducing the observable squeezing. Therefore, it is critical to measure and verify the polarization state of the input pump.
    
    \item \textbf{Waveguide Orientation} To verify proper alignment of the fiber and waveguide axis, second harmonic generation (SHG) is used. A polarization-matched(matched using a Waveplate and PBS) 1064\,nm beam is sent through the same fiber-waveguide assembly, and the 532\,nm SHG output is monitored. Maximizing SHG efficiency provides a practical way to confirm phase-matching and alignment. 
\end{enumerate}

\vspace{0.5em}

\subsubsection{Squeezed State Detection}

The detection of the squeezed state is performed using a balanced homodyne detection scheme. In this method, the squeezed 1064\,nm vacuum field emerging from the PPLN waveguide is combined with a bright 1064\,nm local oscillator (LO) on a 50:50 beamsplitter. The LO is derived from the same Diabolo laser by tapping off a fraction of the 1064\,nm beam using a polarizing beamsplitter. Its spatial mode and polarization are carefully matched to those of the squeezed field using a combination of half-waveplate, PBS, and lenses.\\

To enable quadrature selection, a piezo-actuated mirror (LO phase shifter) in the LO path introduces a variable phase shift. This allows the measurement of arbitrary quadratures of the squeezed state by controlling the relative phase between the LO and the signal field.\\

The two outputs of the beamsplitter are directed to matched low-noise photodiodes (Koheron 1811, labeled PD1 and PD2) via steering mirrors and beam-shaping optics. The outputs of the photodiodes are electronically subtracted to produce a signal proportional to the quadrature amplitude of the squeezed field.\\

To measure squeezing in the time domain, the subtracted signal is sent to a lock-in amplifier of Moku. The subtracted signal is split and mixed to obtain sidebands. Then the DC component is extracted using a low pass filter by setting a corner frequency and analyzed. Scanning the LO phase with a sinusoidal drive on the PZT enables reconstruction of the squeezed state and anti-squeezed state levels.

\paragraph{Potential Limitations in Squeezed State Detection:}
\begin{enumerate}
    \item \textbf{Mode Mismatch:} Efficient interference between the LO and the squeezed beam requires high spatial mode overlap (visibility). Even minor mismatches in beam waist, divergence, or alignment reduce the interference contrast, effectively mixing in vacuum noise and degrading the measurable squeezing. Therefore, precise mode matching using adjustable lens systems and alignment tools is essential.
    
    \item \textbf{Photodiode Gain Imbalance:} Any asymmetry between the gains of the two photodetectors can lead to incomplete subtraction and an artificial increase in the noise floor. This can be corrected through electrical gain balancing or calibration procedures using unsqueezed reference beams.
    
    \item \textbf{Phase Drift and PZT Modulation Frequency:} To distinguish true squeezing from slow drifts in the LO-signal relative phase (caused by mechanical or thermal fluctuations), the PZT must be driven at an appropriate frequency. If the modulation is too slow, it becomes difficult to distinguish quantum noise variation from classical phase noise. A carefully selected modulation frequency can ensure reliable quadrature scanning.
    
    \item \textbf{Technical Noise Contributions (Moku and PD Noise):} The total noise floor must be below the shot noise to observe squeezing. If the electronic noise from the detection electronics (e.g., Moku system) or the dark noise from photodetectors is comparable to or greater than the shot noise, the squeezing will be masked. Hence, the unsqueezed shot noise level must be calibrated and, if necessary, increased by adjusting the LO power to ensure it dominates over technical noise. Full noise characterization of the detectors and electronics is therefore essential.
\end{enumerate}
Our goal was to generate and characterize squeezed vacuum states using the above-described waveguide-based Optical Parametric Amplification (WOPA) setup and ultimately maximize the squeezing level achieved by optimizing the existing setup. Toward this, it is important to address each of the potential limitations addressed above and overcome them. To do so, we carried out and will be carrying out in the future the following series of experiments to overcome these limitations and achieve decent squeezing levels of the order of 4 dB:

\subsection{Polarization Mismatch }

\textbf{Experiment}

1) According to the PPLN waveguide datasheet, the fast axis is oriented vertically. To match this, the input beam entering the fiber coupler was vertically polarized using a half-wave plate (HWP) in conjunction with a polarizing beam splitter (PBS). The optical power of this vertically polarized beam was measured using a calibrated power meter.

2) To characterize the polarization state of light entering the crystal, the fiber was deliberately misaligned at the coupling interface with the waveguide. This ensured that light exiting the fiber could be analyzed independently of the waveguide's internal birefringence.

3) A second PBS was introduced after the aspheric lens, at the output side of the waveguide.

4) The power meter was positioned at the reflected (vertically polarized) port of the PBS. The input fiber coupler mount was then rotated while monitoring the power reading. The angle that maximized the reflected power was taken as the orientation corresponding to maximum vertical polarization at the crystal input.

5) The same procedure was repeated for both 1064\,nm and 532\,nm input beams. The ratio of the power reflected (vertically polarized) to the total transmitted power was taken as a measure of the degree of linear polarization.

\textbf{Results}

\begin{itemize}
    \item 1064\,nm beam: $\sim$81\% linearly polarized in the vertical direction
    \item 532\,nm beam: $\sim$93\% linearly polarized in the vertical direction
\end{itemize}

Since the field was never at 100\% true reflection (vertical polarization) upon rotating the PBS, we infer that the polarization state of the light exiting the fiber is elliptical. Possible causes include:

\begin{itemize}
    \item Residual mode mixing or imperfect alignment at the fiber-waveguide interface
    \item Bending-induced mechanical stress in the fiber, leading to birefringence and polarization rotation
\end{itemize}

\textbf{Future Follow-up}

1) To minimize polarization distortion from the fiber, a new length of polarization-maintaining fiber (P3-1064PM-FC-1) will be cut to multiples of its beat length. The polarization characterization experiment described above will be repeated using this optimized fiber. If significant ellipticity remains, the experimental setup will be modified to reduce fiber bending and mechanical strain.

2) It is also possible that the actual fast axis of the waveguide is not aligned with the nominal vertical direction. Therefore, the polarization angle of the linearly polarized input beam will be systematically varied using a HWP. The polarization orientation that yields the maximum nonlinear gain—either via second harmonic generation (SHG) or spontaneous parametric down-conversion (SPDC)—will be selected for subsequent squeezing experiments. This empirical optimization ensures that the input polarization is correctly aligned with the crystal’s effective nonlinear axis.


\subsection{Waveguide Orientation}

\textbf{Experiment}

To ensure that the fiber coupler and the waveguide are properly aligned for optimal nonlinear interaction, the following measurements were conducted:

\begin{enumerate}
    \item \textbf{Coupling Efficiency under Unphased Conditions:}  
    The waveguide was first held at room temperature, a condition under which quasi-phase-matching is not satisfied and no significant nonlinear interaction (e.g., SHG or SPDC) occurs. Under these conditions, the coupling efficiency of the input beam into the PPLN waveguide was determined. This was done by measuring the power of the beam before the 40$\times$ focusing lens (input to the bare fiber) and comparing it to the output power measured after the aspheric lens (output from the waveguide). The maximum coupling efficiencies achieved through alignment of the bare fiber with the PPLN input facet were as follows:
    \begin{itemize}
        \item 1064\,nm beam: $\sim$36\% coupling efficiency
        \item 532\,nm beam: $\sim$47\% coupling efficiency
    \end{itemize}
    
    \item \textbf{Phase-Matched SHG Measurement:}  
    A 1064\,nm beam was aligned and coupled into the waveguide. The crystal temperature was then tuned to 52.5$^\circ$C to achieve quasi-phase-matching for second harmonic generation (SHG). A dichroic mirror was used to separate the generated 532\,nm light from the fundamental. The SHG output power was recorded, and the normalized conversion efficiency was calculated using the following expression:
    
    \begin{figure}[H]
        \centering
        \includegraphics[height=5cm]{figures/NLG.png}
        \caption{Second harmonic generation in PPLN waveguide \cite{manual}.}
        \label{fig:enter-label}
    \end{figure}
    
    \[
    \text{Normalized Conversion Efficiency}~\left(\frac{\%}{\text{Watt}/\text{cm}^2}\right) = 
    \frac{
        \dfrac{X_{\text{SHG}}~[\mu\text{W}]}{(1 - R_{\text{OUTPUT}})(1 - L_{\text{LOSS}})}
    }{
        \left( \dfrac{Y_{\text{PUMP}}~[\text{mW}]}{(1 - R_{\text{OUTPUT}})(1 - L_{\text{LOSS}})} \right)^2 \times (L~[\text{cm}])^2
    }
    \times 100\%
    \]
    
\end{enumerate}

\textbf{Result}

With an input pump power of 9.5\,mW at 1064\,nm and a measured SHG output power of 123\,$\mu$W at 532\,nm, the normalized SHG conversion efficiency of the waveguide was found to be approximately 60\%/W/cm².

\vspace{0.5em}

\textbf{Future Follow-up}

This method of orienting the waveguide assumes that the polarization and alignment conditions that optimize SHG also correspond to those that maximize spontaneous parametric down-conversion (SPDC), which underlies squeezed light generation. However, this assumption may not hold strictly due to differing phase-matching bandwidths and modal overlaps for SHG and OPA.

To more rigorously optimize the waveguide for squeezing, the nonlinear gain in the SPDC process should be directly maximized. This will be done by launching a high-power 532\,nm pump beam along with a low-power 1064\,nm seed beam into the waveguide. The amplification of the seed 1064\,nm beam through OPA will be measured as a function of pump power and input polarization. The nonlinear gain will be calculated using the expression:

\[
\text{Nonlinear Gain} = \frac{X_{\text{SPDC}}}{Y_{\text{signal}} \cdot W_{532}}
\]

where:
\begin{itemize}
    \item $X_{\text{SPDC}}$ is the amplified power of the 1064\,nm light generated via SPDC,
    \item $Y_{\text{signal}}$ is the input power of the 1064\,nm seed beam,
    \item $W_{532}$ is the input pump power at 532\,nm.
\end{itemize}

  




 

\subsection{Mode Matching}

\textbf{Experiment}

To achieve efficient balanced homodyne detection (BHD), it is essential that the spatial modes of the local oscillator (LO) and the signal (squeezed vacuum) beam are well overlapped. Mode matching was optimized using a beam profiler to align and match the beam waists and divergence parameters of the LO and signal paths. Through iterative adjustments of lens positions and beam steering mirrors, we achieved beam size agreement within approximately 10--20\,$\mu$m.

The quality of mode matching was quantified using a time-domain visibility measurement. This involved monitoring the direct current (DC) outputs of each photodiode while phase drifted. The visibility \( V \) was calculated using the standard formula:

\[
V = \frac{I_{\text{max}} - I_{\text{min}}}{I_{\text{max}} + I_{\text{min}}}
\]

\textbf{Result} \\
The measured BHD visibility was \( V \approx 0.6 \), indicating a moderately good mode overlap between the LO and the signal beams.

\textbf{Future Follow-up} \\
Further optimization of mode-matching optics and strategies will be explored by another SURF student.
\vspace{1em}

\subsection{Photodiode Gain Imbalance}
\textbf{Experiment}

To suppress common-mode noise sources—such as residual frequency noise from the laser, electronics noise, and environmental vibrations—it is essential that the outputs of the two photodetectors are accurately gain-matched prior to subtraction. This ensures that only differential quantum fluctuations-shot noise are retained in the signal.

Gain balancing was performed by first blocking the signal path and directing approximately 0.6\,mW of 1064\,nm LO power onto each photodiode. The subtraction of the AC outputs was implemented digitally using the Moku:Lab’s FIR filter builder, with gains adjusted empirically to minimize residual baseline noise. The optimal configuration was:

\[
\text{North PD gain: } +1.2 \qquad \text{South PD gain: } -0.9
\]

\textbf{Result} \\
\begin{itemize}
    \item At low frequencies, some residual peaks remained in the gain-balanced output, likely due to electronic pickup or incomplete subtraction of technical noise.
    \item At higher frequencies, the output trace was flat, confirming that the detection was shot-noise-limited. This was further validated by comparing the observed noise level to theoretical expectations. The shot noise amplitude spectral density \( S \) was estimated using:
    \[
    S = \left( \sqrt{2 h \nu P} \right) \cdot (QR)(TRI)(R)(\text{Gain}) \approx 1.33 \times 10^{-6}~\frac{V}{\sqrt{\text{Hz}}}
    \]
    where \( h \) is Planck’s constant, \( \nu \) is the optical frequency, \( P \) is the LO power, and the other terms represent detector-specific quantum response, transimpedance gain, reflection coefficients, and amplifier gain. The agreement with theory confirms proper gain balancing and validates the measurement technique.
\end{itemize}

Therefore, for all future squeezing measurements, the signal will be bandpassed at higher frequencies(MHz range) where the trace is shot-noise-limited. This flat noise floor will be used as the reference against which squeezing below shot noise can be quantified.

\begin{figure}[H]
    \centering
    \includegraphics[height=11cm]{figures/BHD_gainb.png}
    \caption{Shot-noise-limited detection after gain balancing. Residual low-frequency peaks are present, while the high-frequency trace is flat and agrees with the theoretical shot noise level.}
    \label{fig:enter-label}
\end{figure}



\subsection{Squeezing Measurement}
\textbf{Experiment}

After confirming mode matching between the signal and local oscillator (LO) beams, and applying gain balancing to the photodetector outputs as described in the previous sections, we proceeded with the generation and measurement of squeezed vacuum states.

\begin{enumerate}
    \item \textbf{Baseline Shot Noise Calibration:}  
    To establish a reference quadrature variance, we blocked the signal arm and directed only the LO beam onto the two photodiodes. The differential signal, obtained by subtracting the gain-balanced outputs, was then band-passed between 2.5--7\,MHz—where the detection is shot-noise limited—and amplified by 5\,dB. The filtered and amplified signal was then routed from the Moku:Lab to the Moku:Go via a physical BNC cable.

    In Moku:Go’s multi-instrument mode, the Lock-In Amplifier was selected. The input signal was digitally split, assigning both mixer inputs to channel In1, operating in AC mode at 10\,Vpp and 0\,dB attenuation. Internal modulation was enabled for phase reference. The output of the mixer was low-pass filtered at 500\,Hz (18\,dB/octave), and the resulting trace was observed on Moku:Go’s oscilloscope.

    \item \textbf{Squeezing Generation:}  
    The 532\,nm pump beam was then unblocked, and the maximum available pump power was coupled into the PPLN waveguide. The pump power measured immediately after the dichroic mirror following the waveguide was approximately 2\,mW. The LO power incident on the balanced photodetectors was maintained at 0.6\,mW.

    \item \textbf{Observation of Squeezing:}  
    Upon enabling the 532\,nm pump, random phase fluctuations between the squeezed vacuum state and the LO were observed. These manifested as modulations in the quadrature noise signal, seen as fluctuations in the lock-in amplifier’s low-pass output. The spontaneous phase drift is attributed to mechanical and thermal fluctuations in the optical paths, causing random sampling of different quadratures of the squeezed state.

    The differential signal (with pump on) was processed in the same way as the shot noise baseline: it was band-passed between 2.5--7\,MHz, amplified by 5\,dB, and routed to Moku:Go for lock-in detection. The observed baseline quadrature variance was approximately 50\,mV under vacuum (unsqueezed) conditions. Upon enabling the pump, modulations around this baseline were observed, corresponding to squeezed and anti-squeezed quadratures.
\end{enumerate}

The levels of squeezing and anti-squeezing, relative to the shot noise baseline, were calculated in decibels using:

\[
\text{Level}_{\text{dB}} = 20 \log_{10} \left( \frac{V_{\text{squeezed/anti-squeezed}}}{V_{\text{vacuum}}} \right)
\]

\textbf{Result} 
 
Using this relation, we observed:
\begin{itemize}
    \item Squeezing: $\sim -0.14$\,dB  
    \item Anti-squeezing: $\sim +0.29$\,dB
\end{itemize}
These values confirm the successful generation and detection of squeezed vacuum states.

\vspace{0.5em}

Following this, the 532\,nm pump power was varied systematically, and the corresponding squeezing and anti-squeezing levels were measured. The pump power values reported on the x-axis of the plot were measured immediately after the dichroic mirror following the waveguide. The resulting trend is shown in the following figure:

\begin{figure}[H]
    \centering
    \includegraphics[height=11cm]{figures/Squeezing_diff_pumpp.png}
   \caption{ Squeezing level at  different pump powers}
    \label{fig:enter-label}
\end{figure}

\subsection{Squeezing Measurement by Sweeping the PZT in the LO Path}

To distinguish true squeezing from slow phase drifts between the signal and local oscillator (LO) beams—caused by mechanical or thermal fluctuations—a piezoelectric transducer (PZT)-mounted mirror was placed in the LO path to enable active phase modulation. This was implemented in three stages:

\begin{enumerate}
    \item Determining the frequency and voltage requirements for achieving a full-wavelength phase sweep.
    \item Characterizing the mechanical response of the PZT by measuring its transfer function and identifying resonant frequencies, after locking out low-frequency phase drifts.
    \item Performing squeezing measurements while continuously driving the PZT.
\end{enumerate}

\textbf{Experiment}

\subsubsection*{Step 1: Characterizing Phase Sweep Range}

A 100\,Hz triangular waveform (0-10 V) was amplified by a factor of 15(in the PZT driver) and applied to the PZT. The same waveform was also fed into Moku:Go as the reference signal. The resulting phase modulation was confirmed via the interference pattern observed at the balanced homodyne detector (BHD ) voltage output in the South PD:

\begin{itemize}
    \item Slightly more than one wavelength of phase sweep for 75\,V input (reference signal was taken by spliting the input therefore power became half)
    \item Approximately two wavelengths for 150\,V input
\end{itemize}

\begin{figure}[H]
    \centering
    \includegraphics[height=11cm]{figures/PZT_75_labelled.png}
    \caption{PZT-induced phase sweep at 75\,V input}
\end{figure}

\begin{figure}[H]
    \centering
    \includegraphics[height=11cm]{figures/PZT_150_labelled.png}
    \caption{PZT-induced phase sweep at 150\,V input}
\end{figure}

This range enables observation of two full squeezing and anti-squeezing cycles within a single triangular sweep, allowing time-domain tomography of the squeezed quadrature.

\subsubsection*{Step 2:  Locking Drift Noise to measure the transfer function}

To characterize the PZT-mounted mirror’s transfer function and avoid mechanical resonances during operation, we first designed a feedback loop that can suppress low-frequency drift noise and actuate the PZT with an external signal. The block diagram of the feedback loop is shown below:

\begin{figure}[H]
    \centering
    \includegraphics[height=6cm]{figures/PZT_loop_detailed.png}
    \label{TF}
    \caption{Feedback loop for drift noise suppression- }
\end{figure}
Here,
\begin{itemize}
 \item E: Error signal (Output from PD1 - PD2).

\item C: Controller (Integrator in Moku PID).

\item P: Plant (PZT including mechanical mirror response).
\item E": Excitation (Sin Wave)
\item S: Sensor.
\end{itemize}

Initially, only the drift noise locking part was implemented as follows-
\begin{itemize}
    \item The error signal (DC difference between North and South photodiodes) was fed into a Controller (integrator in Moku-Pro's PID module).
    \item The controller output was amplified using an SRS-560 preamplifier and the PZT driver with gains of 10\,V/V and 15\,V/V respectively.
\end{itemize}

\begin{figure}[H]
    \centering
    \includegraphics[height=11cm]{figures/MOKU_PID_Phase_Locking.png}
    \caption{Locked state of phase drift using Moku PID + SRS-560 amplifier}
\end{figure}

\begin{figure}[H]
    \centering
    \includegraphics[height=11cm]{figures/Locked_Unlocked_State.png}
    \caption{Transition from locked to unlocked phase state upon disabling the controller}
\end{figure}

Attempts to lock the drift noise using Moku alone were unsuccessful due to insufficient output voltage. However, incorporating the SRS-560 amplifier enabled stable locking, as evidenced by the clean transition between locked and unlocked states.

\subsubsection*{Step 3: Determining the Transfer Function Using Frequency Response Analysis}

To characterize the frequency response of the PZT-mounted mirror and identify any mechanical resonances, we employed frequency-domain analysis under a drift-locked condition.

\textbf{Assumptions:}
\begin{itemize}
    \item \( C(f) \): Transfer function of the controller (integrator)
    \item \( P(f) \): Transfer function of the plant (PZT + mirror)
    \item \( S(f) \): Sensor transfer function (assumed unity: \( S(f) = 1 \))
    \item \( V_{\text{in1}} \): Error signal after applying excitation
    \item \( V_{\text{in2}} \): Combined input to the plant (error + excitation)
\end{itemize}

The open-loop transfer function of the system can be expressed as (see Figure~\ref{TF}):

\[
G_{\text{open}}(f) = \frac{V_{\text{in1}}}{V_{\text{in2}}} = C(f) \cdot P(f) \cdot S(f)
\Rightarrow \log\left( \frac{V_{\text{in1}}}{V_{\text{in2}}} \right) = \log C(f) + \log P(f)
\Rightarrow \log P(f) = \log\left( \frac{V_{\text{in1}}}{V_{\text{in2}}} \right) - \log C(f)
\]

Thus, by measuring \( V_{\text{in1}} \) and \( V_{\text{in2}} \) and subtracting the known contribution of the controller \( C(f) \), we can isolate and compute the plant’s frequency response \( P(f) \).

To implement this, the Moku-Pro’s multi-instrument interface was configured with three PID controllers and a Frequency Response Analyzer (FRA). The setup enabled us to inject known sine wave excitations into the locked feedback loop and compute the plant’s response over a range of frequencies via the above relation.

Configuration in the  Moku-Pro’s multi-instrument interface was as follows:
\begin{itemize}
    \item \textbf{PID 1:} Inputs 1 and 2 received DC outputs from the North and South PDs. These were subtracted to produce an error signal \( V_{\text{in1}} \).
    
    \item \textbf{PID 2:} Input 1 was \( V_{\text{in1}} \). Input 2 was an excitation sine wave from the FRA. The sum, \( V_{\text{in2}} \), was fed to PID 3.
    
    \item \textbf{PID 3:} \( V_{\text{in2}} \) was integrated to act as the controller. The gain-frequency (i.e., the integrator’s zero gain-cross frequency) was tuned to 1\,kHz and 1.175\,kHz in different trials.
    
    \item \textbf{FRA:} Inputs 1 and 2 received \( V_{\text{in1}} \) and \( V_{\text{in2}} \), respectively. The output was a frequency-swept sine wave fed to PID 2. The FRA computed the transfer function magnitude as
    \[
    \log |G(f)| = \log\left( \frac{V_{\text{in1}}}{V_{\text{in2}}} \right),
    \]
    representing the combined frequency response of the controller and plant (PZT system).
\end{itemize}
\begin{figure}[H]
    \centering
    \includegraphics{figures/Amplitude_Bode_Plot.pdf}
    \caption{Amplitude Bode Plot of the PZT system}
     \label{Am}
\end{figure}

\textbf{Remarks on Fig:\ref{Am}:} The initial dip near 10\,Hz is due to temporary loss of lock and not indicative of PZT response. For ideal PZTs, resonance typically occurs in the 100\,kHz range, and amplitude steadily decreases with frequency. Adding a mirror reduces resonant frequencies. In our case, no distinct resonance was observed below 1\,kHz, suggesting this frequency range is safe for driving the PZT.

However, due to stability limits in the controller(drift noise cannot be locked ), we were unable to increase the gain-frequency significantly beyond 1\,kHz, preventing resolution of the PZT response in higher-frequency regions.

\textbf{Result}  

The amplitude Bode plots at 1\,kHz and 1.175\,kHz gain values confirm that the system responds well from 15\,Hz to 1\,kHz. No prominent resonant peaks were observed in this range, suggesting this bandwidth is suitable for phase modulation without significant amplitude distortion.

\subsubsection*{Future Follow-up}

\begin{enumerate}
    \item We will proceed with squeezing measurements by driving the PZT using sinusoidal excitations at 1\,kHz. The low-pass filter corner frequency in the lock-in amplifier can be safely set near 100\,kHz for quadrature demodulation.
    
    \item To resolve the PZT system’s behavior beyond 1\,kHz, we plan to increase the excitation amplitude. This will improve signal-to-noise ratio and allow detection of potential mechanical resonances in the high-frequency region, identifying the maximum usable modulation frequency for phase control.

\end{enumerate}






\subsection{Technical Noise Contributions: Moku-Pro and Photodiode Noise}

To reliably observe squeezing, the total system noise floor must lie below the shot noise level. If electronic noise from the acquisition system (e.g., Moku-Pro) or dark noise from the photodetectors (PDs) is comparable to or exceeds the shot noise, squeezing signatures will be obscured. Therefore, accurate noise characterization of both detection electronics and PDs is essential. Additionally, the unsqueezed shot noise level must be calibrated and, if necessary, increased by adjusting the LO power to ensure it dominates over technical noise.



\subsubsection{ Moku-Pro ADC Input Noise Characterization}

\textbf{Experiment}
To measure the intrinsic input noise of the Moku-Pro, Input 1 was properly terminated using a 50\,$\Omega$ Micro-Circuits BRTM-50+ connector. The Spectrum Analyzer module was configured with the following settings:

\begin{itemize}
    \item Y-axis: Vpp, PSD units
    \item Input range: 400\,mVpp (lowest range to reduce quantization noise)
    \item Input impedance: 50\,$\Omega$ (matched to termination)
    \item Coupling: DC (0–1\,kHz), AC (1\,kHz–300\,MHz)
\end{itemize}

To improve the signal-to-noise ratio in the low-frequency regime, frame averaging was applied. The resolution bandwidth (RBD) was manually tuned across frequency segments to clearly resolve spectral features. These segmented traces were then stitched together to produce a continuous Amplitude Spectral Density (ASD) plot.

\subsubsection{ Photodiode Dark Noise  ASD Characterization}
\textbf{Experiment}

The room was darkened to suppress ambient light. The AC outputs of the North and South PDs were connected to Inputs 1 and 2 of the Moku-Pro, respectively, via BNC cables. The Spectrum Analyzer settings were kept identical to those used for the Moku-Pro noise measurement. Although the PDs have a 33\,$\Omega$ output impedance, a 50\,$\Omega$ input setting was used on the Moku-Pro to ensure impedance matching. RBD and frame averaging were again tuned for optimal spectral resolution.

\textbf{Measurement details:}

Moku-Pro In-1 Acquisition:

    DC: 0–100 Hz: RBW=1.24 Hz, 15 avg,
    100 Hz–1 kHz: RBW=5 Hz, 15 avg 

     AC: 
    1–5 kHz: RBW=19.55 Hz, 31 avg, 
    5–10 kHz: RBW=24.44 Hz, 55 avg,
    10–100 kHz: RBW=439.9 Hz, 5 avg,
    100 kHz–3 MHz: RBW=4.5 kHz,
    3–10 MHz: RBW=20.31 kHz,
    10–50 MHz: RBW=162.5 kHz, 
    50–300 MHz: RBW=325 kHz.

North/South PD Acquisition:

    DC: 0–500 Hz: RBW=1.24 Hz, 15 avg,  
    500 Hz–1 kHz: RBW=8 Hz
     AC: 1–10 kHz: RBW=19.34 Hz, 40 avg, 
    10–50 kHz: RBW=84.63 Hz, 40 avg, 
    50–100 kHz: RBW=104.2 Hz, 23 avg, 
    100–500 kHz: RBW=846.3 Hz, 23 avg, 
    500 kHz–1 MHz: RBW=1.128 kHz, 
    1–300 MHz: RBW=40.62 kHz.
\begin{figure}[H]
    \centering
    \includegraphics[width=0.9\textwidth]{figures/Noise_budget.pdf}
    \caption{
    Amplitude spectral density (ASD) of the Moku-Pro (Input 1) and Photodetectors (North/South PDs). }
     \label{fig:asd_moku_pd}
\end{figure}
 
    

\textbf{Results}

\begin{itemize}
    \item The Moku-Pro input noise floor was measured to be approximately \(1 \times 10^{-7}\,\text{V}/\sqrt{\text{Hz}}\), in good agreement with manufacturer specifications.
    \item A spectral peak was observed in the Moku-Pro noise between 10–100\,kHz.
    \item The PD dark noise spectra showed prominent peaks around 80\,kHz and 100\,MHz, with amplitudes near \(1 \times 10^{-6}\,\text{V}/\sqrt{\text{Hz}}\). 
\end{itemize}

\textbf{Future Follow-Up}

With a local oscillator power of approximately 1\,mW, the shot noise level will exceed both Moku-Pro and PD dark noise levels across all relevant frequency bands. Therefore, this LO power can be safely used in squeezing measurements to ensure that quantum noise dominates over technical noise.




\section{Future Plan Of Action}



The key tasks are summarized below:

\begin{enumerate}
    \item \textbf{Nonlinear Gain Measurement:}  
    Perform power measurements of 1064\,nm and 532\,nm beams before and after key optics to evaluate signal and pump contributions. Use this data to compute nonlinear conversion efficiency.

    \item \textbf{Polarization Optimization:}  
    Optimize input polarization of the combined beams using a half-wave plate. Measure output power as a function of polarization angle to determine the optimum for nonlinear gain.

    \item \textbf{Fiber-Induced Polarization Drift Diagnosis:}  
    Investigate whether polarization drift arises from the optical fiber or the crystal. Replace the fiber with a polarization-maintaining one of equal effective beat length and compare performance.

    \item \textbf{Mode Matching:}  
    Use Gaussian beam propagation tools (e.g., Finesse or ABCD matrix formalism) to model the 1064\,nm signal and LO paths. Align experimentally to achieve $>$95\% overlap at the beamsplitter and verify using beam profiling.

    \item \textbf{PZT Response and Fast Squeezing Measurement:}  
    Characterize the PZT’s mechanical response in the high-frequency domain. Implement fast squeezing measurement using lock-in detection with a corner frequency $>$100\,kHz and 1 kHz sine wave excitation to drive the PZT. calculate loss in the system by fitting the squeezing level vs pump power data for the fast squeezing measurement using the method described in \cite{tse}

    \item \textbf{Loss Characterization:}  
    Estimate expected losses from component datasheets and compare with measured values using power meters. Quantify the impact of optical losses and mode mismatch on observed squeezing levels by comparing with observed total loss obtained in the last step.
\end{enumerate}



\begin{thebibliography}{9}

\bibitem{zetie_mzi}\label{1}
K. P. Zetie, S. F. Adams, and R. M. Tocknell, 
\emph{How does a Mach–Zehnder interferometer work?}, 
Physics Education \textbf{35}(1), 46–48 (2000).

\bibitem{mckenzie}
K. McKenzie, 
\emph{Squeezing in the Audio Gravitational Wave Detection Band}, 
Ph.D. thesis, The Australian National University (2008).

\bibitem{goda}
K. Goda, 
\emph{Development of Techniques for Quantum-Enhanced Laser-Interferometric Gravitational-Wave Detectors}, 
Ph.D. thesis, Massachusetts Institute of Technology (2007).

\bibitem{saleh_teich}
M. C. Teich and B. E. A. Saleh, 
\emph{Squeezed States of Light}, 
Quantum Optics: Journal of the European Physical Society B \textbf{1}, 153–191 (1989); reprinted in \emph{Tutorials in Optics}, ed. D. T. Moore (Optical Society of America, 1992), ch. 3, pp. 29–52.
\bibitem{tse}
M. Tse \textit{et al.},
\emph{Quantum-enhanced advanced LIGO detectors in the era of gravitational-wave astronomy},
Phys. Rev. D \textbf{104}, 062006 (2021).

\bibitem{ligo_website}
LIGO Laboratory, \emph{LIGO Caltech Website}, \url{https://www.ligo.caltech.edu/}
\bibitem{manual}
HC Photonics Corp.
PPLN Waveguide Manual
\end{thebibliography}
\end{document}
\end{document}